\section{Diskussion} \label{sec:Diskussion}
    Aus der vorangegangenen Betrachtung geht hervor, dass alle Forschungsfragen erfolgreich abschließbar waren.

    Aus Forschungsfrage \ref{q:one} ist durch die Konzeption \ref{!!!!!} klargeworden, dass für eine Sicherheitslückenanalyse von Paketen mehrere Funktionalitäten notwendig sind.
    Diese sind in den funktionalen Anforderungen \ref{f:one} und \ref{f:two} aufgefasst.
    Es handelt sich um das Einladen einer Schwachstellendatenbasis, für welche die \ac{CVE}-Daten gewählt wurden, sowie die Über\-prüfung eines Paketes mittels Abgleich dessen Bezeichnung und dieser Schwachstellendatenbasis auf Über\-ein\-stimmungen.

    Die Forschungsfrage \ref{q:two} ist aufbauend auf Forschungsfrage \ref{q:one} eine Schwachstellenanalyse auf einem ganzen Repository und somit auch in der Konzeption aufgezeigt.
    Deshalb werden hier auch mehr funktionale Anforderungen angebracht.
    \\
    Diese sind die funktionalen Anforderungen \ref{f:one}, \ref{f:three}, \ref{f:four} und \ref{f:five}.
    Somit ist diese Forschungsfrage auch in der ersten Implementierung gelöst worden.
    Für eine Analyse eines ganzen Repositories muss nach Clonen und Extraktion des Abhängigkeitsbaums des Repositories für seine Abhängigkeiten auf einer Schwachstellendatenbank der Abhängigkeitsbaum mit Schwachstellendaten angereichert werden.

    Forschungsfrage \ref{q:three} handelt vom Rückgabedatentyp und der weiteren Nutzbarkeit von Resultatdaten.
    Hier wurde durch die Lösung der nicht-funktionalen Anforderung \ref{nf:four} die Rückgabe im JSON-LD-Format genutzt.

    Forschungsfrage \ref{q:four} ist durch die Konzeption und Implementierung des Projektes beantwortet worden.
    Hier ist die funktionale Anforderung \ref{f:seven} und die nichtfuntkionale Anforderung \ref{nf:five} zugehörig.
    Letztendlich wurde also eine mit Docker-Compose containerisierte Applikation entwickelt, welche aus Applikations- sowie Datenbankcontainer besteht.
    Diese hat mit ASP.NET und zuerst LiteDB und dann MySQL jeweils eine aktuelle und weit verbreitete Technologien als Grundgerüst genutzt.

    Forschungsfrage \ref{q:five} war der Grund für eine zweite Iteration der Implementierung.
    Hier wurde mit den nichtfunktionalen Anforderungen \ref{nf:one}, \ref{nf:two} und \ref{nf:three} eine obere Grenze von 5 ms für Einzelpaketsuchen, die Skalierbarkeit der Anwendung und eine Dokumentation der Anwendung gefordert.
    Hierbei ist die obere Grenze der Paketsuche die Ursache der zweiten Implementierung.
    \\
    Die funktionale Anforderung einer Aktualisierungsfunktion (\ref{f:six}) ist auch hier zugehörig für eine bessere Nutzbarkeit der Applikation.
    Es wurde letztendlich mit einer indexierten MySQL-Datenbank eine Suche von unter 5 ms für eine Einzelpaketsuche erreicht.
    Die Skalierbarkeit wiederrum wurde durch den Einsatz der MySQL-Datenbank und der Implementierung des Webservice mit ASP.NET umgesetzt.
    \\
    Eine Dokumentation ist durch die Implementierung der Anwendung als \ac{API} mit einer Swagger-Endpunktbeschreibung geschehen.

    Es gab unter anderem verschiedene Auffälligkeiten die sich während der Entwicklung dieser Arbeit aufgetan haben.
    Darunter, dass in 2014 das Datenmodell der CVE-Daten von der vierten auf die fünfte Version \href{https://github.com/CVEProject/cvelistV5}{github.com/CVEProject/cvelistV5} verändert hat.
    Außerdem haben manche, sehr alte Einträge, keine Paketbezeichnung und sind somit für eine Analyse auf der Basis dieser nicht verwendbar.
    Hierfür mussten diese leeren Bezeichnungen gefüllt werden, wobei die Wahl auf ein \glqq n/a\grqq~für \textit{not available} fiel, da es keine Möglichkeit gibt diese nachträglich automatisiert herauszufinden.

    Diese Arbeit beschränkt sich in ihrer Implementierung auf die Analyse von npm-Projekten, da bei der Extraktion des Abhängigkeitsbaumes dieser Projekte auf die nativen Funktionen von npm zurückgegriffen werden konnte.
    Weiterhin wurden ausschließlich \ac{CVE}-Einträge als einzige Schwachstellendatenquelle genutzt.
    Dies führt dazu, dass Schwachstellen eventuell bestehen, aber nicht -- in der für die Analyse genutzten -- Datenbasis vorhanden sind, womit diese unentdeckt bleiben.
    Somit sollten also zukünftig verschiedene Datenquellen für Schwachstellen genutzt werden können.
    Weiterhin ist hier klar eine Abhängigkeit vom \ac{CVE}-Repository für dessen Daten vorhanden damit eventuell eine Suche nach Alternativen zur Absicherung der Datenbasis vonnöten ist.
    Es sind außerdem keine Sicherheitslücken in der \ac{API} selbst auszuschließen.

    Fortführend von dieser Arbeit können nun verschiedene weitere Betrachtungen vorgenommen werden.
    Darunter die Verteilung von Schwachstellen auf Versions- oder Commit-Basis oder der zeitliche Verlauf von Schwachstellendaten in Repositories.
    Weiterhin können Unterschiede zwischen Projekttypen, Frameworks oder Programmiersprachen bei Abhängigkeiten betrachtet werden.
    Zusätzlich ist es möglich, Dopplungen von Abhängigkeiten gleicher und verschiedener Versionen in einem einzelnen Projekt zu untersuchen.

    In dieser Arbeit wurden auch verschiedene Themen nicht adressiert, welche durchaus einen Schwerpunkt in der Schwachstellenanalyse von Repositories bilden.
    Darunter auch welche Sicherheitslücken wirklich letztendlich ausgenutzt werden können, also welcher Sicherheitskritischer Code wirklich ausgeführt wird.
    Das wurde zum Beispiel von Amir M. Mir et al im Paper \glqq On the Effect of Transitivity and Granularity on Vulnerability Propagation in the Maven Ecosystem\grqq$~$für das Maven-Ökosystem betrachtet.
    Hier wurde herausgefunden, dass zwar ein Drittel der Pakete bei Betrachtung aller Abhängigkeiten Schwachstellen aufweisen, allerdings nur $1\%$ aller Pakete tatsächlich erreichbaren schwachstellenbehafteten Code enthalten.
    Daraus resultierend wurde vorgeschlagen nur eine bestimmte Tiefe des Abhängigkeitsbaumes zu analysieren, um Rechenzeiten zu verringern.
    \\
    In der Arbeit selbst wurde außerdem kein Abhängigkeitsbaum selbst aufgestellt, welcher die Vorgabe von Abhängigkeitsbäumen durch das genutzte Framework notwendig macht.
    Alternativen spielen auch eine große Rolle bei Entscheidung auf Ersetzen eines Paketes.
    Diese aufzuzeigen bringt dem Nutzer letztendlich nach der Entscheidungsfindung den größten Nutzen.
    Viele Pakete bekommen allerdings während ihrer Lebenszeit Aktualisierungen welche manche Sicherheitslücken schließt.
    Hier einen Hinweis anzubringen, ob ein Paket Hoffnung auf einen solchen Patch hat oder veraltet ist wird kann mit Betrachtung der Alternativen dieses mit abgehandelt werden.
    Im größeren Stil kann auch eine Analyse auf die meist verwendeten Sicherheitskritischen Softwarepakete in bestimmten Ökosystemen angebracht werden. 
