\subsubsection{Datenbank} \label{sec:ExperimenteDB}
    Die Wahl der Datenbank hat einen großen Einfluss auf die Laufzeit von Anfragen auf diese.
    Aus der gegebenen Einbindung einer dateibasierten Datenbanklösung in ASP.NET -- LiteDB -- wurde diese auch zuerst als persistenter Datenspeicher gewählt.
    Aufgrund der Suchzeiten von LiteDB bei einzelnen Paketen, die zwischen $2,5$ und $7,2$ Sekunden liegen, erschien die Suche nach einer Alternative für bessere Ergebnisse angeraten.
    Dies wird deutlich an den in Tabellen \ref{tabularx:LessPackagesThenDbFiles} \& \ref{tabularx:MorePackagesThenDbFiles} aufgezeigten Laufzeiten trotz Verbesserung mit der Umsetzung einer Pipeline für die Suche.

    LiteDB ist eine NoSQL-Datenbank und eröffnete so die Möglichkeit der Lösung des Performance-Problems in einer relationalen Datenbank.
    Dazu wurde MySQL als \textit{open-source} und \textit{free-use} Version gewählt.
    Diese unterscheidet sich grundsätzlich durch Indexierung der jeweiligen Einträge sowie dem höheren Maß an nutzbaren Ressourcen durch das laufen auf einem separaten Server.
