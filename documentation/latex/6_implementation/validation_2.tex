\subsection{Validierung V2} \label{sec:Vali2}
    Die unter \ref{sec:Vali1} \nameref{sec:Vali1} bereits erfüllten funktionalen und nichtfunktionalen Anforderungen sind weiterhin erfüllt, da an deren Implementierung keine Veränderungen stattfanden.
    Die noch nicht als erfolgreich gekennzeichnete \texttt{funktionale} Anforderung wird in der V2 nachfolgend neu betrachten.

    \begin{description}
        \item[(6) Aktualisierung der Datenbank] siehe \hyperref[f:six]{\underline{hier}} \hfill \\
            Wurde erfolgreich umgesetzt.
            \\
            Über den hinzugefügten Endpunkt kann die MySQL-Datenbank aktualisiert werden.
    \end{description}

    \noindent Unter \ref{sec:Vali1} \nameref{sec:Vali1} sind folgende \texttt{nicht-funktionale} Anforderungen noch weiter zu validieren.

    \begin{description}
        \item[(1) Dauer der Paketsuche (einzelne Abfrage) unter 5ms] siehe \hyperref[nf:one]{\underline{hier}} \hfill \\
            Wurde erfolgreich umgesetzt.
            \\
            Unter dem Abschnitt \ref{sec:MySQL_Indexierung} ist die Auswertung der Messung mit einer indizierten Datenbanktabelle einzusehen.
            Daraus ist erkenntlich, daß die Vorgabe der 5ms mit den durchschnittlich erreichten $0,7$ms erreicht wurde.
        \item[(2) Skalierbarkeit der Anwendung] siehe \hyperref[nf:two]{\underline{hier}} \hfill \\
            Wurde erfolgreich umgesetzt.
            \\
            Dank der mehrfachen zeitgleichen Zugriffsmöglichkeit auf einer MySQL-Datenbank ist dieser Punkt umgesetzt.
            Die einzige Limitierung liegt in der Hardware, auf welcher der Docker-Container läuft.
    \end{description}
