\subsection{Implementation V2} \label{sec:Implementation2}
    Aus den Ergebnissen der Experimente, aus Kapitel \ref{sec:ExperimenteDB} und \ref{sec:MySQL_Indexierung}, werden Anpassungen an der Implementation vorgenommen.
    Folgende Komponenten wurden für die zweite Implementation der \ac{API} genutzt:
    \begin{itemize}
        \item Framework ASP.NET (C\#)
        \item Datenbank MySQL
        \item Docker-Compose zur Container-Orchestration
    \end{itemize}

    Weiterhin gab es eine Anpassung am Containernetzwerk, die MySQL Datenbank hat nun ihren eigenen Container, da diese nicht in ASP.NET eingebettet ist.
    Außerdem sind folgende Anpassungen an den Controllern vorgenommen worden:
    \begin{itemize}
        \item DbController \label{api_controller:three}
            \begin{itemize}
                \item CheckRawDir-Endpunkt entfällt siehe GitControlller api/Git/PullCveAndConvert Endpunkt
                \item ConvertRawDirToDb-Endpunkt entfällt
                \item Update-Endpunkt hinzugefügt \\
                    Mit diesem Endpunkt werden die lokalen CVE-Daten aktualisiert ohne den gesamten Datensatz erneut zu laden.
                    \\
                    \textbf{OK} bei einem Update mit Rückgabe der Anzahl aktualisierter und eingefügter CVE-Einträge
                    \\
                    \textbf{Bad Request} bei einem Aufgetretenen Fehler
                \item GetFullTextFromCveNumber-Endpunkt hinzugefügt \\
                    Durch diesen Endpunkt erhäklt man bei Übergabe einer CVE-Nummer alle vorliegende Daten zu dieser.
                    \\
                    \textbf{OK} bei Fund der CVE-Nummer mit den vorliegenden Daten im Antwort-Body.
                    \\
                    \textbf{No Content} bei keinem Eintrag zur angegebenen CVE-Nummer. 
                    \\
                    \textbf{Bad Request} bei fehlerfahter übergebener Query.
                \item CheckGuid-Endpunkt hinzugefügt \\
                    Hier kann man durch angabe einer Guid das lokale Vorhandensein eines vorher heruntergeladenen Repositories überprüfen.
                    \\
                    \textbf{OK} falls das Repository mit der passenden Guid vorhanden ist mit dem gesamten DB-Datensatz dieses Repositories.
                    \\
                    \textbf{Not Found} falls das Repository nicht lokal vorliegt.
            \end{itemize}

        \item GitController \label{api_controller:one}
            \begin{itemize}
                \item Clone zu CloneRepo umbenannt
                \item PullCveAndConvert (Endpunkte des ersten Controllers sind nun hier) hinzugefügt
            \end{itemize}
        \item MySqlConnectionController \label{api_controller_2:five}
            \begin{itemize}
                \item CheckReachable-Endpunkt hinzugefügt
            \end{itemize}
        \item ViewController \label{api_controller:four}
            \begin{itemize}
                \item Json-Ld-Endpunkt entfällt
                \item CveResult-Endpunkt hinzugefügt
                \item NodePackageResult-Endpunkt hinzugefügt
            \end{itemize}
           
    \end{itemize}