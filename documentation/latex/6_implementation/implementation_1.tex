\subsection{Implementation V1} \label{sec:Implementation1}
    Folgende Komponenten wurden für die erste Implementation der \ac{API} genutzt:
    \begin{itemize}
        \item Framework ASP.NET (C\#)
        \item Datenbank LiteDB
        \item Docker-Compose zur Container-Orchestration
    \end{itemize}

    \noindent Die genutzte Datenbank LiteDB wurde folgend genutzt:
    \begin{itemize}
        \item Speicherung aller \ac{CVE}-Daten
        \item Je Jahr der vorhandenen \ac{CVE}-Daten wurde eine Datenbankdatei angelegt
    \end{itemize}

    \noindent Folgende Datenmodelle wurden zur Nutzung der Daten in der Datenbank und/oder zur internen Verarbeitung angelegt:
    \begin{itemize}
        \item CVEcomp, für CVE-Complete \\
            Dieses Modell ist eine komplette Representation eines \ac{CVE}-Eintrags.
            Für dieses Modell sind viele weitere Klassen notwendig um mit dem \textit{JSONSerializer} den kompletten Datensatz umwandeln und anschließend in die interne Datenbank einfügen zu können.
            Es befinden sich als Klassen alle weiteren Subelemente der \ac{JSON}-Datei für die korrekte Verarbeitung im Programm.
        \item CveResult \\
            Dieses Modell dient der Rückgabe von \ac{CVE}-Daten bei der Einzelpaket- oder Mehrpaketsuche auf der Datenbasis.
            Hier wird also ein Objekt oder eine Liste dieses Modells zurückgegeben.
        \item JsonLdObject \\
            Mittels diese Modells wird das Verbinden des \ac{JSON-LD} \textit{$@context$} mit dem \textit{$data$} Teil einer Response ermöglicht.
        % \item PackageForApi %! SIEHE V2
        % \item RepoObject %! SIEHE V2
        \item $[$Enum$]$ ProjectType \\
            Dieses Enum wurde eingeführt, damit bei Nutzung des Repository-Analyze-Endpunkts nicht der Projekttyp als String (zum Beispiel NodeJS) sondern ein Ganzzahlwert anstatt dessen genutzt werden kann.
            Die Entscheidung dafür fiel, damit mögliche Schreibfehler vom Endnutzer ausgeschlossen werden können.
    \end{itemize}

    \noindent Es wurden außerdem folgende Controller zur Umsetzung der funktionalen Anforderungen implementiert:
    \begin{itemize}
        \item DbController \label{api_controller:three}\\
            Dieser Controller dient der Befüllung der Datenbank mit \ac{CVE}-Daten sowie der Nutzung dieser für kleinere Analysen. Dies sind zum Beispiel einzelne Pakete oder Listen von Paketen.
            \begin{itemize}
                \item CheckRawDir-Endpunkt \\
                    Dieser Get-Endpunkt dient zur Überprüfung, ob die Roh-Daten für die \ac{CVE}-LiteDB-Datenbank bereits vollständig Herunterladen wurden.
                    \\
                    \textbf{OK} bei Vorhandensein
                    \\
                    \textbf{No Content} bei nicht Vorhandensein der Datenbasis
                \item ConvertRawDirToDb-Endpunkt \\
                    Dieser Endpunkt dient der Konvertierung des heruntergeladenen \ac{CVE}-Daten-Repositories in die intern nutzbaren LiteDB Datenbankdateien.
                    Dies ist für die Vermeidung der Volltextsuche auf den \ac{JSON}-Dateien des \ac{CVE}-Repositories zwingend notwendig.
                    Dabei auffällig ist, dass manche Paketbezeichnungen leer und nicht nachträglich auffindbar sind.
                    Um diese Bezeichnungen nicht leer zu lassen wurde sich für die Bezeichnung \glqq n/a\grqq$~$für \glqq not available\grqq$~$entschieden.
                    \\
                    \textbf{OK} wird immer zurückgegeben.
                \item checkSinglePackage-Endpunkt \\
                    Durch diesen Post-Endpunkt wird durch einen übergebenen Paketnamen und ggf. dessen Version nach \ac{CVE}-Einträgen zu diesem gesucht.
                    Falls es diese gibt werden sie als Liste zurückgegeben.
                    \\
                    \textbf{OK} bei Vorhandensein mit Rückgabe der Liste
                    \\
                    \textbf{No Content} bei nicht Vorhandensein ohne weitere Rückgabe.
                \item checkPackageList-Endpunkt \\
                    Durch diesen Post-Endpunkt werden durch eine übergebene Liste eines Tupels aus Paketnamen und Version nach \ac{CVE}-Einträgen zu diesen gesucht.
                    Falls es diese gibt werden sie als Liste zurückgegeben.
                    \\
                    \textbf{OK} bei Vorhandensein mit Rückgabe der \ac{CVE}-Daten-Liste
                    \\
                    \textbf{No Content} bei nicht Vorhandensein ohne weitere Rückgabe
            \end{itemize}

        \item DependeciesController \label{api_controller:two} \\
            Dieser Controller dient der Extraktion des Abhängigkeitsbaumes von Repositories und der Analyse dieser.
            Dabei bedeutet Analyse eine Untersuchung auf \ac{CVE}-Daten für jegliche im Repository enthaltene Pakete.
            \begin{itemize}
                \item ExtractTree-Endpunkt \\
                    Bei diesem Get-Endpunkt wird je nach Projekttyp, zum Beispiel NodeJs, der Abhängigkeitsbaum extrahiert.
                    \\
                    \textbf{OK} mit dem Abhängigkeitsbäumen für alle Abhängigkeiten des Repositories im Response-Body
                    \\
                    \textbf{Bad Request} bei nicht Vorhandensein des angegebenen Projekttyps
                \item ExtractAndAnalyzeTree-Endpunkt \\
                    Bei diesem Get-Endpunkt handelt es sich um eine Erweiterung des ExtractTree-Endpunkts.
                    Es der wird zusätzlich für jedes Paket im Repository und in dessen Abhängigkeitsbaum eine Suche auf der CVE-Datenbasis durchgeführt und in der jeweiligen Stelle im Abhängigkeitsbaum das Attribut isCveTracked wahr gesetzt wird.
                    \\
                    \textbf{OK} mit dem Analysierten Abhängigkeitsbäumen für alle Abhängigkeiten des Repositories im Response-Body
                    \\
                    \textbf{229 Keine Schwachstelle gefunden} bei keiner gefundenen Schwachstelle in allen Abhängigkeitsbäumen
                    \\
                    \textbf{Bad Request} bei nicht Vorhandensein des angegebenen Projekttyps
            \end{itemize}

        \item GitController \label{api_controller:one} \\
            Da die \ac{CVE}-Daten sowie die zu untersuchenden Repositories ihre Datenbasis in GitHub\textsuperscript{\cite{GITHUB}} haben, werden diese durch diesen Controller gecloned.
            \begin{itemize}
                \item Clone-Endpunkt \\
                    Bei diesem Post-Endpunkt wird das CVE-Daten-Repository bzw. das zu analysierende Repository aus GitHub heruntergeladen.
                    Übergeben wird hier ein Tupel aus Strings, welches die URL und ein Tag beinhaltet sowie einen Bool-Wert, der angibt, ob es sich um CVE-Daten oder ein Analyse-Repository handelt.
                    \\
                    \textbf{OK} bei erfolgreichem Herunterladen
                    \\
                    \textbf{Bad Request} bei nicht erfolgreichem Herunterladen
            \end{itemize}

        \item ViewController \label{api_controller:four}\\
            In diesem Controller werden als HTML alle \ac{JSON-LD} Beschreibungen -- siehe $@context$ -- zurückgegeben.
            \begin{itemize}
                \item json-ld-Endpunkt \\
                    In diesem Get-Endpunkt wird die JSON-LD-Definition als sichtbare HTML-Daten dargestellt.
                    \\
                    \textbf{OK} mit den HTML-Daten als Inhalt des Response-Bodys
            \end{itemize}
    \end{itemize}

    \noindent Um diese Applikation jetzt unabhängig vom Betriebssystem des Endgerätes zu machen, wurde eine docker-compose-Datei für die Containerisierung erstellt.
    Dieses referenzierte ein einzelnes Dockerfile mit dem Container der API.
    Ein eigenständiger Datenbankcontainer ist aufgrund der Eigenart von LiteDB als dateibasierte Datenbank nicht notwendig.
