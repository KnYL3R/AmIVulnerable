\documentclass[10pt,a4paper,twoside]{article}
\usepackage[top=1.5in,bottom=1in,left=1.125in,right=1.125in]{geometry}
\usepackage{amsmath}
\usepackage{amssymb}
\usepackage{amsthm}
\usepackage[center]{caption}
\usepackage{booktabs}
\usepackage{xcolor}
\usepackage{fancyhdr}
\usepackage{graphicx}
\usepackage{latexsym}
\usepackage{lipsum}
\usepackage{longtable}
\usepackage{multirow}
\usepackage{multicol}
\usepackage{tasks}
\usepackage[singlespacing]{setspace}
\usepackage[ngerman]{babel}
\usepackage{wrapfig}
\usepackage{pgfplots}

% Biblatex
\usepackage{csquotes}
\usepackage[
    backend=biber,
    style=numeric,
    natbib=true,
    sorting=none,
    block=nbpar,
    backref=true
]{biblatex}
\addbibresource{lib/bib.bib}
\usepackage{hyperref}

% If you want to break on URL numbers
\setcounter{biburlnumpenalty}{9000}
% If you want to break on URL lower case letters
\setcounter{biburllcpenalty}{9000}
% If you want to break on URL UPPER CASE letters
\setcounter{biburlucpenalty}{9000}

\pagestyle{myheadings}
% \setlength{\parindent}{0.8in}

\usepackage{float}

% Graph--------------------------------------------------------------------------------------------
\usepackage{graphics,graphicx}
\usepackage{calc}
\usepackage{tikz}
\usetikzlibrary{decorations.markings}
\tikzstyle{vertex}=[circle, draw, inner sep=2pt, minimum size=4pt]
\newcommand{\vertex}{\node[vertex]}
\newcounter{Angle}

\renewcommand{\baselinestretch}{1.25}

% ???
\newcommand*{\QEDA}{\hfill\ensuremath{\large{\lozenge}}}
\newcommand*{\QEDAa}{\hfill\ensuremath{\square}}

% Centering table of contents and list of table
% \usepackage{tocloft}
% \renewcommand{\contentsname}{\hfill\bfseries\Large Inhaltsverzeichnis \hfill}
% \renewcommand{\cftaftertoctitle}{\hfill}

\usepackage[explicit]{titlesec}
\setcounter{secnumdepth}{3}
\setcounter{tocdepth}{3}

\definecolor{DarkGreen}{RGB}{3,103,16}
\renewcommand{\thefootnote}{\textcolor{DarkGreen}{\arabic{footnote}}}

\usepackage[bottom]{footmisc}

% Variablen ---------------------------------------------------------------------------------------
    \newcommand{\autorFirstNameT}{Tim}
    \newcommand{\autorFirstNameK}{Konstantin}
    \newcommand{\autorFamilyNameT}{Kretzschmar}
    \newcommand{\autorFamilyNameK}{Blechschmidt}
    % Kürzere Referenzierung
% Variablen ---------------------------------------------------------------------------------------

% Eigenschaften einstellen
\hypersetup{
    pdftitle={},
    pdfauthor={\autorFirstNameT \autorFirstNameK} {\autorFamilyNameT \autorFamilyNameK},
    pdfcreator={\autorFirstNameT \autorFirstNameK} {\autorFamilyNameT \autorFamilyNameK},
    pdfkeywords={AmIVulnerable} {CVE} {Both} {\autorFamilyNameT} {\autorFamilyNameK},
    pdfnewwindow = true,
    colorlinks,
    pdfpagelabels,
    pdfstartview = FitH,
    bookmarksopen = true,
    bookmarksnumbered = true,
    linkcolor = black,
    urlcolor = blue,
    plainpages = false,
    hypertexnames = false,
    citecolor = blue
}

\begin{document}
    \nocite{*}

    \phantomsection
    \addcontentsline{toc}{section}{Inhaltsverzeichnis}
    \tableofcontents\label{Inhaltsverzeichnis}

    \newpage

    % Lorem, ipsum dolor sit amet consectetur adipisicing elit. Molestias aut, repellat ipsum facere voluptate dicta obcaecati deserunt nobis suscipit eaque?
    % \begin{wrapfigure}{l}{0.5\textwidth}
    %     \centering
    %     \begin{tikzpicture}
    %         \begin{axis}[
    %             ybar,
    %             bar width=0.5cm,
    %             width=0.48\textwidth,
    %             height=6cm,
    %             ymin=0,
    %             xlabel={$x$},
    %             ylabel={Werte},
    %             xtick={1,2,3,4,5},
    %             xticklabels={1, 2, 3, 4, 5},
    %             xtick=data,
    %             nodes near coords,
    %             nodes near coords style={/pgf/number format/.cd,fixed zerofill,precision=1},
    %             nodes near coords align={vertical},
    %             ]
    %             \addplot coordinates {(1,1) (2,34) (3,23) (4,123) (5,84)};
    %             \addplot[smooth, red, mark=*] coordinates { 
    %                 (0, {(1 + 34 + 23 + 123 + 84) / 5})
    %                 (6, {(1 + 34 + 23 + 123 + 84) / 5})
    %             };
    %         \end{axis}
    %     \end{tikzpicture}
    %     \caption{Balkendiagramm}
    % \end{wrapfigure}
    % Lorem, ipsum dolor sit amet consectetur adipisicing elit. Molestias aut, repellat ipsum facere voluptate dicta obcaecati deserunt nobis suscipit eaque?
    % Lorem, ipsum dolor sit amet consectetur adipisicing elit. Molestias aut, repellat ipsum facere voluptate dicta obcaecati deserunt nobis suscipit eaque?
    % Lorem, ipsum dolor sit amet consectetur adipisicing elit. Molestias aut, repellat ipsum facere voluptate dicta obcaecati deserunt nobis suscipit eaque?
    % Lorem, ipsum dolor sit amet consectetur adipisicing elit. Molestias aut, repellat ipsum facere voluptate dicta obcaecati deserunt nobis suscipit eaque?
    % Lorem, ipsum dolor sit amet consectetur adipisicing elit. Molestias aut, repellat ipsum facere voluptate dicta obcaecati deserunt nobis suscipit eaque?
    % Lorem, ipsum dolor sit amet consectetur adipisicing elit. Molestias aut, repellat ipsum facere voluptate dicta obcaecati deserunt nobis suscipit eaque?
    % Lorem, ipsum dolor sit amet consectetur adipisicing elit. Molestias aut, repellat ipsum facere voluptate dicta obcaecati deserunt nobis suscipit eaque?
    % Lorem, ipsum dolor sit amet consectetur adipisicing elit. Molestias aut, repellat ipsum facere voluptate dicta obcaecati deserunt nobis suscipit eaque?

    \newpage
    \section{Einleitung} \label{sec:Einleitung}
    (1)
    als sub sections:\\
    Problemstellung
    Motivation
    Vorgehen (zur Lösung)
    \section{Recherche} \label{sec:Recherche}
    \subsection{CVE} \label{subsec:CVE}
CVE, oder Common Vulnerabilities and Exposures \cite{}
    \subsection{Umgang mit CVE} \label{subsec:Umgang_mit_CVE}
Umgang mit CVE
    \subsection{Referenzen} \label{subsec:Referenzen}
    Referenzen und \ac{CVE}-Nutzungen
    \subsubsection{NIST-API} \label{subsubsec:NIST_API}
    NIST-API
    \subsubsection{DEPENDA-BOT} \label{subsubsec:DEPENDA_BOT}
    DEPENDA-BOT (GITHUB)
    \subsubsection{Weitere} \label{subsubsec:Weitere}
    Weitere
    \section{Umsetzung} \label{sec:Umsetzung}
    (3)
    (sub)
    Technologiestack
        ASP.NET
        DOCKER
        LiteDB (NOSQL) erst
        MySQL (SQL) Ausblick
    API
        Modells
            PUML (Klassen)
            Interaktion und Zusammenhang der Klassen/Fktn
        Controller
            Auflistung und Erklärung (Swagger screenshot) in Paragraphs
            Views (JSON-LD def von CVE-Result, JSON-LD von NodePackageResult)
        DB-Anbindung: LiteDB braucht DB-Controller
        
    \section{Analyse} \label{sec:Analyse}
    (4)
    (sub)
    Laufzeitmessungen nach Datenbankerstellung (etwa 1 std rechnen)
        JSON-Datei Suche VS LiteDB suche (60 min vs 8 s): Rechnung nach wie vielen Suchanfragen sich das ganze gelohnt hat (nach etwa einer)
    Laufzeitanalyse Mono-Suche und Pipeline-Suche auf LiteDB
        Wie ist LiteDB-Suche zu verschnellern
        Vergleich der Messungen
        Erklärung der Pipeline (Diagramme) (Wie und Warum, LiteDB ist File-Basier und deswegen kann auf mehreren Dateien gesucht werden falls genug threads zur verfügung stehen)
    Ausblick
        Ziel: schneller als NIST API sein (so und so viele Sekunden übers Netz im vergleich zu so und so vielen sekunden über unsere API)
        Um Ziel zu erreichen: Andere Datenbank verwenden um Lesezugriffe optimieren

    \section{Conclusion} \label{sec:Conclusion}
    \subsection{Vision} \label{subsec:Vision}
Vision
    Validieren ob Ziel erreicht ist, dass
        ein Node-JS-Projekt komplett analysisert (alle Pakete entnommen werden) und nach Schwachstellen untersucht werden
        das zurückgegebene Format der API sinnvoll und struckturiert ist
        Wurde sich an programmiertechnische Standards gehalten
            Models sind eigene Bibliothek (losgelöst von der API, können weiterverwendet werden getrennt und weiterentwickelt werden)
            Controller liegen im Controller-Namespace
            Sinnvolle Trennung von Endpunkten in Controller mit http-Signatur
            Routennamen der Endpunkte in natürlicher Sprache/Pfaden
    \input{subsections/nächste_schritte.tex}
            

    \newpage
    \phantomsection
    \printbibheading[title={Quellen}]
    \addcontentsline{toc}{section}{Quellen}
    \printbibliography[type=book, heading=subbibliography, title={Literatur}]
    % \newpage
    % \printbibliography[type=online, heading=subbibliography, title={Links}]
    % \newpage
    % \printbibliography[type=article, heading=subbibliography, title={Zeitschriften}]

    \newpage
    \phantomsection
    \addcontentsline{toc}{section}{Abbildungen}
    \listoffigures
    \newpage
    \phantomsection
    \addcontentsline{toc}{section}{Tabellen}
    \listoftables

\end{document}
