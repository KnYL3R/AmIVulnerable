\documentclass[10pt,a4paper,twoside]{article}
\usepackage[top=1.5in,bottom=1in,left=1.125in,right=1.125in]{geometry}
\usepackage{amsmath}
\usepackage{amssymb}
\usepackage{amsthm}
\usepackage[center]{caption}
\usepackage{booktabs}
\usepackage{xcolor}
\usepackage{fancyhdr}
\usepackage{graphicx}
\usepackage{latexsym}
\usepackage{lipsum}
\usepackage{longtable}
\usepackage{multirow}
\usepackage{multicol}
\usepackage{tasks}
\usepackage[singlespacing]{setspace}
\usepackage[ngerman]{babel}
\usepackage{wrapfig}
\usepackage{pgfplots}

% Biblatex
\usepackage{csquotes}
\usepackage[
    backend=biber,
    style=numeric,
    natbib=true,
    sorting=none,
    block=nbpar,
    backref=true
]{biblatex}
\addbibresource{lib/bib.bib}
\usepackage{hyperref}

% If you want to break on URL numbers
\setcounter{biburlnumpenalty}{9000}
% If you want to break on URL lower case letters
\setcounter{biburllcpenalty}{9000}
% If you want to break on URL UPPER CASE letters
\setcounter{biburlucpenalty}{9000}

\pagestyle{myheadings}
% \setlength{\parindent}{0.8in}

\usepackage{float}

% Graph--------------------------------------------------------------------------------------------
\usepackage{graphics,graphicx}
\usepackage{calc}
\usepackage{tikz}
\usetikzlibrary{decorations.markings}
\tikzstyle{vertex}=[circle, draw, inner sep=2pt, minimum size=4pt]
\newcommand{\vertex}{\node[vertex]}
\newcounter{Angle}

\renewcommand{\baselinestretch}{1.25}

% ???
\newcommand*{\QEDA}{\hfill\ensuremath{\large{\lozenge}}}
\newcommand*{\QEDAa}{\hfill\ensuremath{\square}}

% Centering table of contents and list of table
% \usepackage{tocloft}
% \renewcommand{\contentsname}{\hfill\bfseries\Large Inhaltsverzeichnis \hfill}
% \renewcommand{\cftaftertoctitle}{\hfill}

\usepackage[explicit]{titlesec}
\setcounter{secnumdepth}{3}
\setcounter{tocdepth}{3}

\definecolor{DarkGreen}{RGB}{3,103,16}
\renewcommand{\thefootnote}{\textcolor{DarkGreen}{\arabic{footnote}}}

\usepackage[bottom]{footmisc}

% Variablen ---------------------------------------------------------------------------------------
    \newcommand{\autorFirstNameT}{Tim}
    \newcommand{\autorFirstNameK}{Konstantin}
    \newcommand{\autorFamilyNameT}{Kretzschmar}
    \newcommand{\autorFamilyNameK}{Blechschmidt}
    % Kürzere Referenzierung
% Variablen ---------------------------------------------------------------------------------------

% Eigenschaften einstellen
\hypersetup{
    pdftitle={},
    pdfauthor={\autorFirstNameT \autorFirstNameK} {\autorFamilyNameT \autorFamilyNameK},
    pdfcreator={\autorFirstNameT \autorFirstNameK} {\autorFamilyNameT \autorFamilyNameK},
    pdfkeywords={AmIVulnerable} {CVE} {Both} {\autorFamilyNameT} {\autorFamilyNameK},
    pdfnewwindow = true,
    colorlinks,
    pdfpagelabels,
    pdfstartview = FitH,
    bookmarksopen = true,
    bookmarksnumbered = true,
    linkcolor = black,
    urlcolor = blue,
    plainpages = false,
    hypertexnames = false,
    citecolor = blue
}

\begin{document}
    \nocite{*}

    \phantomsection
    \addcontentsline{toc}{section}{Inhaltsverzeichnis}
    \tableofcontents\label{Inhaltsverzeichnis}

    \newpage

    % Lorem, ipsum dolor sit amet consectetur adipisicing elit. Molestias aut, repellat ipsum facere voluptate dicta obcaecati deserunt nobis suscipit eaque?
    % \begin{wrapfigure}{l}{0.5\textwidth}
    %     \centering
    %     \begin{tikzpicture}
    %         \begin{axis}[
    %             ybar,
    %             bar width=0.5cm,
    %             width=0.48\textwidth,
    %             height=6cm,
    %             ymin=0,
    %             xlabel={$x$},
    %             ylabel={Werte},
    %             xtick={1,2,3,4,5},
    %             xticklabels={1, 2, 3, 4, 5},
    %             xtick=data,
    %             nodes near coords,
    %             nodes near coords style={/pgf/number format/.cd,fixed zerofill,precision=1},
    %             nodes near coords align={vertical},
    %             ]
    %             \addplot coordinates {(1,1) (2,34) (3,23) (4,123) (5,84)};
    %             \addplot[smooth, red, mark=*] coordinates { 
    %                 (0, {(1 + 34 + 23 + 123 + 84) / 5})
    %                 (6, {(1 + 34 + 23 + 123 + 84) / 5})
    %             };
    %         \end{axis}
    %     \end{tikzpicture}
    %     \caption{Balkendiagramm}
    % \end{wrapfigure}
    % Lorem, ipsum dolor sit amet consectetur adipisicing elit. Molestias aut, repellat ipsum facere voluptate dicta obcaecati deserunt nobis suscipit eaque?
    % Lorem, ipsum dolor sit amet consectetur adipisicing elit. Molestias aut, repellat ipsum facere voluptate dicta obcaecati deserunt nobis suscipit eaque?
    % Lorem, ipsum dolor sit amet consectetur adipisicing elit. Molestias aut, repellat ipsum facere voluptate dicta obcaecati deserunt nobis suscipit eaque?
    % Lorem, ipsum dolor sit amet consectetur adipisicing elit. Molestias aut, repellat ipsum facere voluptate dicta obcaecati deserunt nobis suscipit eaque?
    % Lorem, ipsum dolor sit amet consectetur adipisicing elit. Molestias aut, repellat ipsum facere voluptate dicta obcaecati deserunt nobis suscipit eaque?
    % Lorem, ipsum dolor sit amet consectetur adipisicing elit. Molestias aut, repellat ipsum facere voluptate dicta obcaecati deserunt nobis suscipit eaque?
    % Lorem, ipsum dolor sit amet consectetur adipisicing elit. Molestias aut, repellat ipsum facere voluptate dicta obcaecati deserunt nobis suscipit eaque?
    % Lorem, ipsum dolor sit amet consectetur adipisicing elit. Molestias aut, repellat ipsum facere voluptate dicta obcaecati deserunt nobis suscipit eaque?

    \newpage

    \section*{Abstract} \label{sec:abstract}
    Das Entwickeln von Softwarelösungen ohne Bibliotheken, Frameworks oder externe Module ist heutzutage nicht mehr denkbar.
    Abhängigkeiten bestimmen Funktionalität, Effizienz und Sicherheit der jeweiligen Softwarelösung und sind so ein sensibler Punkt der Softwareentwicklung.
    Allerdings ist die Verwaltung dieser Abhängigkeiten mit ihren, teils transitiv vererbten, Sicherheitslücken mit steigender Zahl immer komplexer.
    Um nun also eine fundierte Entscheidung über eine Abhängigkeit treffen zu können muss die mangelnde Transparenz der Abhängigkeitsstruktur von Softwarelösungen berichtigt werden.
    \\
    Diese Arbeit stellt die Entwicklung einer \ac{API} vor, die es ermöglicht transitive Vererbungen von Abhängigkeiten darzustellen.
    Laufzeitmessungen und Vergleiche durch diese mit anderen Tools zeigen die Effizienz und Wirksamkeit der entwickelten Lösung.
    Ziel einer solchen Lösung ist es, das Verständnis und Management von Abhängigkeiten in der Softwareentwicklung zu verbessern, was letztendlich zu einer verbesserten Sicherheit und Effizienz von Softwarelösungen führen soll.

    \section{Motivation} \label{subsec:Motivation}
    Beim Entwickeln von Softwarelösungen gibt es viele Herausforderungen und Probleme. 
    Diese werden durch viele bereits vorhandene Softwarepakete, wie etwa Bibliotheken oder Frameworks, bewältigt.
    Die Nutzung frei verfügbarer Softwarepakete ist deshalb im Arbeitsalltag gang und gäbe.
    Freiwillige oder Hobby-Programmierer ermöglichen mit ihrem Einsatz, dass weltweit die Entwicklung neuer Software sowohl im kommerziellen als auch privaten und öffentlichen Bereich vereinfacht, vereinheitlicht und beschleunigt wird.
    Dank der Konkurrenz freier Pakete, zum Beispiel anhand ihrer Nutzungszahl, gestaltet sich dort ein Wettbewerb, der Pakete (a) an Bedeutung verlieren lässt bzw. (b) soweit verbessert, dass ihre Funktionen und Benutzbarkeit anschließend überzeugen konnten.

    Neben der Funktionalität besteht ein anderer essentieller Aspekt in der Sicherheit.
    Eben jene muss sich bei jedem Paket separat und gekapselt gesehen auf einem solchem Niveau befinden, dass ihre Verwendung keine fahrlässig Gefahr darstellt.
    Dies beginnt bei zu kurzen Schlüssellängen und endet bei komplexen Programmen mit verschiedenen Angriffsschwachstellen.
    Aber die Aufgabe, für jedes verwendete Paket einzeln die Sicherheitslücken nachzulesen oder für eine Paketsammlung nachzuvollziehen, ist selbst mit dem vorhandenen Angeboten zeitaufwendig und ressourcenintensiv -- schließlich werden so personelle Kräfte und Rechenkapazitäten gebunden.

    Daher besteht ein dringender Bedarf an einem Werkzeug, dass Entwicklern und Managern eine transparente und umfassende Analyse der Abhängigkeiten auf Sicherheitslücken ganzer Projekte bietet.
    Ein solches Automatisierungswerkzeug für die Analyse von Paketen trägt somit einer frühzeitigen Erkennung von Sicherheitslücken und -risiken bei und reduziert Zeitaufwand den der Nutzer.
    Damit kann die Softwarequalität verbessert werden, indem erkannt werden kann, ob eine genutzte Softwarekomponente aktualisiert oder ausgetauscht werden muss.
    Hieraus ergeben sich folgende Forschungsfragen (FF):
    \begin{enumerate}[label=\textbf{FF-\Roman*}, leftmargin=1.75cm]
        \item \textbf{Welche Funktionen sind notwendig um ein Paket auf Sicherheitslücken zu untersuchen?}\label{q:one}
        \item \textbf{Welche Funktionen sind notwendig um ein Repository auf Sicherheitslücken durch Abhängigkeiten zu untersuchen?}\label{q:two}
        \item \textbf{Wie sind die Resultat-Daten für den Endnutzer sowie Maschinen zu strukturieren, damit jene ohne weitere aufwändige Aufbereitung zu verarbeiten sind?}\label{q:three}
        \item \textbf{Wie ist ein Schwachstellen-Analyse-Tool in Betracht auf transitive Anhängig\-keiten von Repositories zu implementieren?}\label{q:four}
        \item \textbf{Wie kann ein solches Tool leistungsstärker sowie laufzeiteffizienter werden?}\label{q:five}
    \end{enumerate}
    Insgesamt bringt solch ein Werkzeug Unternehmen verschiedene Vorteile, wie in der Softwareentwicklung, darunter Transparenz, Qualität sowie Sicherheit.
    Die Gesamtsicherheit von Applikationen sowie auch Unternehmen kann durch nun sichtbare Schwachstellen in tiefen Ebenen der Softwareabhängigkeit verbessert werden.
    Dieses Werkzeug hilft also bei der Entscheidungsfindung über externe Softwarekomponenten sowie bei der Aufdeckung von Sicherheitsrisiken.
    
    \section{Definitionen} \label{sec:Definitionen}
\subsection{CVE} \label{subsec:CVE}
CVE, oder Common Vulnerabilities and Exposures \cite{}
    \section{Andere Arbeiten} \label{sec:Andere}
Vor der Konzeption der Softwarelösung einer Dependency-API muss erst ein Blick auf die bisherige Forschung und Entwicklung in diesem Bereich geworfen werden.
Eine Vielzahl von Arbeiten beschäftigt sich mit der Herausforderung, Abhängigkeiten in Softwarelösungen zu verwalten und potenzielle Sicherheitsrisiken zu identifizieren.
Dabei reicht das Spektrum von manuellen Ansätzen bis hin zu automatisierten Tools und Plattformen, die Entwicklern dabei helfen, die Qualität und Sicherheit ihrer Software zu verbessern.
In diesem Artikel werden einige der bekanntesten Softwarelösungen betrachtet, die für diese Zwecke entwickelt wurden, sowie deren Ansätze und Funktionen analysiert.
Es soll ein Überblick über den aktuellen Stand der Technik und mögliche Herausforderungen entstehen um Herausforderungen, denen sich Entwickler bei der Verwaltung von Abhängigkeiten und der Gewährleistung der Sicherheit in ihren Projekten gegenübersehen besser einzuschätzen.
\subsubsection{\ac{NIST}-\ac{API}} \label{subsubsec:NIST_API}
    \ac{NIST}-\ac{API}
\subsection{Github Dependa Bot} \label{sec:Dependa}

\subsection{Snyk} \label{sec:Snyk}
https://docs.github.com/en/code-security/getting-started/github-security-features
\subsubsection{OWASP Dependency-Check} \label{sec:OWASP-Dependency-Check}
Das Software-Composition-Analysis-Tool Dependency Check von der OWASP Foundation analysiert die Codebasis auf bekannte Schwachstellen.
Dabei werden auf Common-Platform-Enumeration-Kennungen (CPE) für die genutzten Abhängigkeiten geprüft und falls vorhanden ein Bericht mit zugehöriger \ac{CVE}-Nummer erstellt.
\\ \\
Die durch die automatischen Analysen erstellten Berichte enthalten nicht nur die jeweiligen Schwachstelle selbst, sondern auch Maßnahmen zum Schließen jener.
\subsection{Mend.io} \label{sec:Mend-io}
https://docs.github.com/en/code-security/getting-started/github-security-features
    \section{Konzept} \label{sec:Konzept}
    Im Rahmen dieser Arbeit soll eine API zur Verfügung gestellt werden, die es ermöglicht transitive Vererbungen von Abhängigkeiten darzustellen.
    Dabei sollen einzelne Pakete, Listen von Paketen und ganze Projekte selbst analysiert werden können.
    Abhängigkeiten werden durch Extraktion eines Abhängigkeitsbaumes erkannt und einzeln gegen Schwachstellendaten geprüft.
    Der Nutzer, bzw. Maschinen sollen letztendlich ein weiterverwendbares klares Datenformat erhalten in welchem betroffene Abhängigkeiten und Ihre Schwachstellen vermerkt sind.
    \subsection{Forschungsfragen} \label{sec:Forschungsfragen}
    Aus den funktionalen und nichtfunktionalen Anforderungen ergeben sich folgende Forschungsfragen für diese Arbeit:
    \\ \\
    Forschungsfrage \ref{q:one}, welche aus den funktionalen Anforderungen \ref{f:one} und \ref{f:two} hervorkommt, betrachtet die Analyse von einzelnen oder Listen von Abhängigkeiten auf Basis der eingeladenen Schwachstellendaten.
    Je nachdem, ob eine Menge an Abhängigkeiten oder eine Einzelne gesucht werden muss müssen hier verschiedene Algorithmen implementiert werden um eine effiziente Ergebnisfindung zu gestalten.
    \\ \\
    Forschungsfrage \ref{q:two} handelt von den zu implementierenden Funktionalitäten für Repositories und geschieht aufbauen auf Forschungsfrage \ref{q:one}.
    Deshalb sind die folgenden funktionalen Anforderungen \ref{f:one}, \ref{f:three}, \ref{f:four} und \ref{f:five} zu betrachten.
    Dazu muss zuerst betrachtet werden, welche Ausgangsdaten vorliegen.
    Zusätzlich auch die zu analysierenden Repositories bzw. Abhängigkeiten und die genutzten Schwachstellendaten.
    \\ \\
    Forschungsfrage \ref{q:three} definiert den Rückgabetyp der Daten.
    Vor allem die nichtfunktionale Anforderung \ref{nf:four}, der Rückgabe im JSON-LD-Format, ist Ausschlaggebend.
    Hier müssen demnach sinnvolle Rückgabedaten für den Endnutzer identifiziert sowie eine Definition dieser vorgenommen werden.
    \\ \\
    Forschungsfrage \ref{q:four} behandelt die Umsetung der API selbst.
    Es ist die Umsetzung der Funktionalen Anforderungen unter der Betrachtung der nicht funktionalen Anforderungen.
    Vor allem nichtfunktionale Anforderung \ref{nf:five}, die Nutzung von Etablierten Technologien, hat hier ihren Schwerpunkt.
    Hierfür wird eine Programmiersprache mit zugehörigem Framework ausgewählt sowie das Endpunktdokumentations- und Testmöglichkeitstool.
    Es muss sich in diesem Umfeld auch auf eine Programmiersprache im Sinne der zu untersuchenden Projekte entschieden werden um den Umfang dieser Arbeit möglichst genau zu halten.
    Der entstandene Service muss, wie durch funktionale Anforderung \ref{f:seven} bestimmt, weiterhin im Container-Kontext nutzbar sein.
    \\ \\
    Forschungsfrage \ref{q:five} beschäftigt sich mit der Optimierung der API.
    Funktionale Anforderung \ref{f:six} und nichtfunktionale Anforderungen \ref{nf:one}, \ref{nf:two} und teils auch \ref{nf:three} sind hier zugehörig.
    Hier müssen algorithmische sowie technologische Verbesserungen in Leistung oder Laufzeit aufgezeigt und mögliche Lösungswege aufgezeigt oder implementiert werden.
    Besonders gewählte Datenbankformate werden hier thematisiert.

    
    
    \subsection{Funktionalitäten} \label{sec:Funktionalitäten}
    Bei der Umsetzung dieser API sind verschidene Funktionalitäten zu implementieren.
    \begin{enumerate}
        \item Einladen und Konvertieren der Schwachstellendaten und persistente Speicherung dieser in einer internen Datenbank.
        \item Überprüfung von Paketen auf Sicherheitslücken mittels Abgleich zur internen Schachstellendatenbank.
        \item Clonen eines Repositories von Github.
        \item Überprüfung von allen Abhängigkeiten eines Repositories mittels Abgleich zur internen Schachstellendatenbank.
        \item Rückgabe eines Abhängigkeitsbaums für alle Abhängigkeiten mit Sicherheitslücken. 
    \end{enumerate}
    Diese Funktionalitäten setzten teils mehrere interne Abläufe voraus und beinhalten mehrere Endpunkte der API.

    \subsection{Besondere Merkmale} \label{sec:Besondere Merkmale}
Durch die große Menge an CVE-Daten, die als Schachstellen-Suchbasis dient, können diese nicht vor jeder Anfrage erneut abgerufen werden.
Somit ist es notwendig diese abzuspeichern und ggf. nur zu aktualisieren.
\\ \\
Da das Rückgabeformat der API JSON ist und bei Analyse eines ganzen Repositories hier Abhängigkeitsbäume aufgezeigt werden ist die Rückgabe dann teils schlecht händisch zu durchsuchen.
Deshalb ist es empfohlen ein Frontendservice an den entsprechenden Endpunkt anzuknüpfen.
    \subsection{Erwartete Ergebnisse} \label{sec:Erwartete Ergebnisse}
    Folgende Ergebnisse sind im Rahmen dieser Arbeit zu erreichen:
    \begin{enumerate}
        \item \textbf{Entwicklung einer funktionsfähigen API} \\
            Diese soll performant und möglichst schmal in einem Containercontext nutzbar sein und alle -- von den Features -- notwenigen Endpunkte beinhalten.
        \item \textbf{Analyse auf Sicherheitslücken} \\
            Die entstehende API soll Sicherheitslücken in Abhängigkeiten ausfindig machen und diese in einem weiterverwendbaren Format zurückliefern.
        \item \textbf{Darstellung der detektierten Schwachstellenpakete inklusive Abhängigkeiten als Baum} \\
            Bei Analyse eines Repositories sollen alle Abhängigkeiten dieses mit dessen Abhängigkeiten aufgezeigt werden.
        \item \textbf{Entscheidungsunterstützung bei Abhängigkeitsveränderungen} \\
            Die entstehende API soll dem Nutzer letztendlich dabei helfen, eine Entscheidung über das Weiterverwenden oder Austauschen bzw. Aktualisieren einer Abhängigkeit zu fällen.
    \end{enumerate}

    \subsection{Architektur} \label{sec:Architektur}
    \subsubsection{Architektur V1} \label{sec:ArchitekturV1}
    Für die Umsetzung der Anforderungen sind verschiedene Komponenten in der Umsetzung der \ac{API} nötig:
    \begin{enumerate}
        \item \textbf{Framework} \label{arch_1}\\
            Um eine \ac{API} zu entwickeln, muss ein passendes Framework genutzt werden.
            Dadurch, dass keine Notwendigkeit für ein Frontend besteht gibt es hier wenig Eingrenzungen.
            \\
            Aus Erfahrungsgründen wurde das C\# Framework ASP.NET gewählt.
            Mit diesem lassen sich unter anderem \ac{API}-Services bauen.
            Mittels der für das .NET Framework explizit vorhandenen IDE \glqq Visual Studio\grqq~bestehen auch native Debugging-Möglichkeiten.
            Durch die große und aktive Community von ASP.NET ist dieses Framework sehr gut dokumentiert und es existieren viele Open-Source-Bibliotheken.
        \item \textbf{Datenbank} \label{arch_2}\\
            In der Datenbank sind alle \ac{CVE}-Daten zu persistieren, welche zu durchsuchen sind.
            Diese Daten dienen als Grundlage der Identifizierung von Schwachstellen in Paketen innerhalb von Projekten.
            Wichtig sind hier schnelle Lesezugriffe, da diese beim Abfragen der Datenbank die größte Laufzeiteinsparung bringen.

            Durch die Wahl des Frameworks auf ASP.NET ist es möglich einen \textit{embedded NoSQL-Document-Store} -- LiteDB zu nutzen. % TODO: https://www.litedb.org/
            Diese ist leicht intern nutzbar und muss als \textit{file-based DB} nicht separat explizit gestartet und verwaltet werden, womit diese auch im Container der \ac{API} mit enthalten ist.
        \item \textbf{Controller} \label{arch_3}\\
            Controller einer \ac{API} nehmen HTTP-Anfragen entgegen und reagieren darauf.
            Hier wird die Hauptaufgabe der \ac{API} geschehen, da alle Funktionalitäten, sei es Datenbankabfragen, Klonen eines zu untersuchenden Repositories oder die Untersuchung dieses, in einem solchen implementiert oder aufgerufen werden müssen.

            Notwendig sind hier vier Controller.
            (\hyperref[api_controller:one]{1}) Es muss ein Git-Controller zum nutzen von \ac{CVE}-Daten sowie zum Erhalt von zu analysierenden Repositories entstehen.
            In diesem sind Endpunkte zum clonen des \ac{CVE}-Daten-Repositories sowie zum clonen des Analyse-Repositories zu implementieren. % TODO: clone CVE-DATEN-REPO-LINK
            \\
            Weiterhin ist (\hyperref[api_controller:two]{1}) ein Controller für Abhängigkeiten nötig, in dem man aus dem zu analysierenden Repositoriy den Abhängigkeitsbaum extrahiert sowie diesen mit Schwachstellendaten anreichert.
            \\
            Für die Untersuchung einzelner Pakete und Listen dieser ist ein weiterer (\hyperref[api_controller:three]{3}) Endpunkt zu implementieren.
            In diesem ist auch die Update-Funktion der Datenbasis hinzuzufügen.
            \\
            Weiterhin muss in jedem Endpunkt (\hyperref[api_controller:four]{4}) bei korrekter Antwort ein Context mitgeliefert werden, damit der gelieferte Inhalt so durch JSON-LD zu interpretieren ist.
            Ebenfalls sind durch einen Controller die Rückgabedaten zu dokumentieren.
            Dazu ist zwischen Softwarepaketen und \ac{CVE}-Einträgen zu unterscheiden.
        \item \textbf{Datenmodelle} \label{arch_4}\\
            Um Daten korrekt in die Datenbank einzufügen, um ein Resultat-JSON zu erzeugen oder die Paketliste intern zu verarbeiten -- dazu sind Datenmodelle nötig.
        \item \textbf{Konvertierung von und in JSON} \label{arch_5}\\
            Beim Einlesen der \ac{CVE}-Daten in die Datenbank ist eine Konvertierung vom vorhandenen JSON-Format in Einträge der Datenbank vorzunehmen.
            Die aus der Datenbank genutzten, durch den Controller verarbeiteten, Daten müssen nun schließlich im JSON-Format dem Benutzer übermittelt werden.
            \\
            Dies muss in den jeweiligen Controllern geschehen.
            Damit die Daten besser weiterverwendbar sind muss zusätzlich ein Kontext \glqq @context\grqq~hinzugefügt werden.
        \item \textbf{Container} \\
            Um die \ac{API} unabhängig von der Umgebung und möglichst speicheraufwandsschmal zu nutzen muss die Anwendung \glqq Containerisiert\grqq~werden.

            Für den Bau der Containers wird Docker-Compose genutzt.
    \end{enumerate}
    Weiterhin muss das JSON-LD-Format für die Rückgabedaten definiert werden (Forschungsfrage \ref{three}).

    \subsubsection{Architektur V2} \label{sec:Architektur}
    Aus den Ergebnissen des Experiments, aus Kapitel \ref{...}, werden folgende Anpassungen an der Achitektur vorgenommen:
    \begin{enumerate}
        \item \textbf{Datenbank} \\
            Nutzung der SQL-Datenbank MySQL anstatt der eingebetteten No-SQL-Datenbank LiteDB.
    \end{enumerate}

    \textcolor{red}{
        5. bleibt Architektur, 6. API (kann umbenannt werden), darunter 5 Unterpunkte: 1. V1, 2. Verifikation V1 (fktnl, nicht fktnl Anforderungen), 3. Experimente, 4. V2, 5. Verifikation V2 (Unterschiede zu V1 betrachten und)
    }

    \section{Implementation} \label{sec:Implementation}
Vor der Implementation ist die Entscheidung der Auswahl der Einzelkomponenten, welche im Abschnitt \ref{Architektur} vorgestellt wurden, zu treffen.
\\ \\
Zuerst ist die Programmiersprache sowie das genutzte Framework zu definieren.
Aus Erfahrungsgründen wurde das C\# Framework ASP.NET gewählt.
Mit diesem lassen sich unter anderem API-Services bauen.
ASP.NET hat mit Visual Studio auch direkt eine Entwicklungsumgebung sowie Debugging Möglichkeiten.
Weiterhin gibt es die Möglichkeit die eingebette NoSQL-Datenbak LiteDB zu nutzen.
Durch die große und aktive Community von ASP.NET ist dieses Framework sehr gut dokumentiert und es existieren viele Open-Source-Bibliotheken.
\subsection{Experimente} \label{sec:Experimente}
Experimente
    \section{Verifikation} \label{sec:Verifikation}

    \section{Diskussion} \label{sec:Diskussion}
    Aus der vorangegangenen Betrachtung geht hervor, dass alle Forschungsfragen erfolgreich abschließbar waren.

    Aus Forschungsfrage \ref{q:one} ist durch die Architektur \ref{sec:ArchitekturV1} klargeworden, dass für eine Sicherheitslückenanalyse von Paketen mehrere Funktionalitäten notwendig sind.
    Diese sind in den funktionalen Anforderungen \ref{f:one} und \ref{f:two} aufgefasst.
    Es handelt sich um das Einladen einer Schwachstellendatenbasis, für welche die \ac{CVE}-Daten gewählt wurden, sowie die Über\-prüfung eines Paketes mittels Abgleich dessen Bezeichnung und dieser Schwachstellendatenbasis auf Über\-ein\-stimmungen.

    Die Forschungsfrage \ref{q:two} ist aufbauend auf Forschungsfrage \ref{q:one} eine Schwachstellenanalyse auf einem ganzen Repository und somit auch in der Konzeption aufgezeigt.
    Deshalb werden hier auch mehr funktionale Anforderungen angebracht.
    \\
    Diese sind die funktionalen Anforderungen \ref{f:one}, \ref{f:three}, \ref{f:four} und \ref{f:five}.
    Somit ist diese Forschungsfrage auch in der ersten Implementierung gelöst worden.
    Für eine Analyse eines ganzen Repositories muss nach Clonen und Extraktion des Abhängigkeitsbaums des Repositories für seine Abhängigkeiten auf einer Schwachstellendatenbank der Abhängigkeitsbaum mit Schwachstellendaten angereichert werden.

    Forschungsfrage \ref{q:three} handelt vom Rückgabedatentyp und der weiteren Nutzbarkeit von Resultatdaten.
    Hier wurde durch die Lösung der nichtfunktionalen Anforderung \ref{nf:four} die Rückgabe im JSON-LD-Format genutzt.

    Forschungsfrage \ref{q:four} ist durch die Konzeption und Implementierung des Projektes beantwortet worden.
    Hier ist die funktionale Anforderung \ref{f:seven} und die nichtfuntkionale Anforderung \ref{nf:five} zugehörig.
    Letztendlich wurde also eine mit Docker-Compose containerisierte Applikation entwickelt, welche aus Applikations- sowie Datenbankcontainer besteht.
    Diese hat mit ASP.NET und zuerst LiteDB und dann MySQL jeweils eine aktuelle und weit verbreitete Technologien als Grundgerüst genutzt.

    Forschungsfrage \ref{q:five} war der Grund für eine zweite Iteration der Implementierung.
    Hier wurde mit den nichtfunktionalen Anforderungen \ref{nf:one}, \ref{nf:two} und \ref{nf:three} eine obere Grenze von 5 ms für Einzelpaketsuchen, die Skalierbarkeit der Anwendung und eine Dokumentation der Anwendung gefordert.
    Hierbei ist die obere Grenze der Paketsuche die Ursache der zweiten Implementierung.
    \\
    Die funktionale Anforderung einer Aktualisierungsfunktion (\ref{f:six}) ist auch hier zugehörig für eine bessere Nutzbarkeit der Applikation.
    Es wurde letztendlich mit einer indexierten MySQL-Datenbank eine Suche von unter 5 ms für eine Einzelpaketsuche erreicht.
    Die Skalierbarkeit wiederrum wurde durch den Einsatz der MySQL-Datenbank und der Implementierung des Webservice mit ASP.NET umgesetzt.
    \\
    Eine Dokumentation ist durch die Implementierung der Anwendung als \ac{API} mit einer Swagger-Endpunktbeschreibung geschehen.

    Es gab unter anderem verschiedene Auffälligkeiten die sich während der Entwicklung dieser Arbeit aufgetan haben.
    Darunter, dass in 2014 das Datenmodell der CVE-Daten von der vierten auf die fünfte Version \href{https://github.com/CVEProject/cvelistV5}{github.com/CVEProject/cvelistV5} verändert hat.
    Außerdem haben manche, sehr alte Einträge, keine Paketbezeichnung und sind somit für eine Analyse auf der Basis dieser nicht verwendbar.
    Hierfür mussten diese leeren Bezeichnungen gefüllt werden, wobei die Wahl auf ein \glqq n/a\grqq~für \textit{not available} fiel, da es keine Möglichkeit gibt diese nachträglich automatisiert herauszufinden.

    Diese Arbeit beschränkt sich in ihrer Implementierung auf die Analyse von npm-Projekten, da bei der Extraktion des Abhängigkeitsbaumes dieser Projekte auf die nativen Funktionen von npm zurückgegriffen werden konnte.
    Weiterhin wurden ausschließlich \ac{CVE}-Einträge als einzige Schwachstellendatenquelle genutzt.
    Dies führt dazu, dass Schwachstellen eventuell bestehen, aber nicht -- in der für die Analyse genutzten -- Datenbasis vorhanden sind, womit diese unentdeckt bleiben.
    Somit sollten also zukünftig verschiedene Datenquellen für Schwachstellen genutzt werden können.
    Weiterhin ist hier klar eine Abhängigkeit vom \ac{CVE}-Repository für dessen Daten vorhanden damit eventuell eine Suche nach Alternativen zur Absicherung der Datenbasis vonnöten ist.
    Es sind außerdem keine Sicherheitslücken in der \ac{API} selbst auszuschließen.

    Fortführend von dieser Arbeit können nun verschiedene weitere Betrachtungen vorgenommen werden.
    Darunter die Verteilung von Schwachstellen auf Versions- oder Commit-Basis oder der zeitliche Verlauf von Schwachstellendaten in Repositories.
    Weiterhin können Unterschiede zwischen Projekttypen, Frameworks oder Programmiersprachen bei Abhängigkeiten betrachtet werden.
    Zusätzlich ist es möglich, mehrfaches Vorkommen von Abhängigkeiten gleicher und verschiedener Versionen in einem einzelnen Projekt zu untersuchen.

    In dieser Arbeit wurden auch verschiedene Themen nicht adressiert, welche durchaus einen Schwerpunkt in der Schwachstellenanalyse von Repositories bilden.
    Darunter auch welche Sicherheitslücken wirklich letztendlich ausgenutzt werden können, also welcher Sicherheitskritischer Code wirklich ausgeführt wird.
    Das wurde zum Beispiel von Amir M. Mir et al im Paper \glqq On the Effect of Transitivity and Granularity on Vulnerability Propagation in the Maven Ecosystem\grqq\textsuperscript{\cite{article:OnTheEffect10123571}} für das Maven-Ökosystem betrachtet.
    Hier wurde herausgefunden, dass zwar ein Drittel der Pakete bei Betrachtung aller Abhängigkeiten Schwachstellen aufweisen, allerdings nur $1$ \% aller Pakete tatsächlich erreichbaren schwachstellenbehafteten Code enthalten.
    Daraus resultierend wurde vorgeschlagen nur eine bestimmte Tiefe des Abhängigkeitsbaumes zu analysieren, um Rechenzeiten zu verringern.
    \\
    In der Arbeit selbst wurde außerdem kein Abhängigkeitsbaum selbst aufgestellt, welcher die Vorgabe von Abhängigkeitsbäumen durch das genutzte Framework notwendig macht.
    Alternativen spielen auch eine große Rolle bei Entscheidung auf Ersetzen eines Paketes.
    Diese aufzuzeigen bringt dem Nutzer letztendlich nach der Entscheidungsfindung den größten Nutzen.
    Viele Pakete bekommen allerdings während ihrer Lebenszeit Aktualisierungen welche manche Sicherheitslücken schließt.
    Hier einen Hinweis anzubringen, ob ein Paket Hoffnung auf einen solchen Patch hat oder veraltet ist wird kann mit Betrachtung der Alternativen dieses mit abgehandelt werden.
    Im größeren Stil kann auch eine Analyse auf die meist verwendeten Sicherheitskritischen Softwarepakete in bestimmten Ökosystemen angebracht werden. 

    \section{Zusammenfassung} \label{sec:Zusammenfassung}
    Die in der \nameref{subsec:Motivation} aufgelisteten Forschungsfragen können wie folgt nach Bearbeitung des Auftrages beantwortet werden:
    \begin{description}
        \item[\ref{q:one}]\hfill \\
            Um die Pakete zu untersuchen ist es notwendig, die \ac{CVE}-Daten in eine interne Datenbank umzuwandeln, da das Suchen auf den Rohdaten -- den \ac{CVE}-\ac{JSON}-Dateien -- zu zeitaufwendig ist.
            Dazu wurde schlussendlich eine MySQL-Datenbank ausgewählt, die dank der Indizierung einer Spalte die gewünschten Leistungsmerkmale aufweist.
        \item[\ref{q:two}]\hfill \\
            In der umgesetzten Version ist es dank der nativen Unterstützung von npm gelungen, die Abhängigkeiten der Pakte als \ac{JSON} zu extrahieren und dann intern so weiterzuverarbeiten, dass eine logische Repräsentation erfolgen kann.
        \item[\ref{q:three}]\hfill \\
            Die Lösung dieser Frage geschah über die einheitliche Verwendung von \ac{JSON} respektive \ac{JSON-LD} bei der Rückgabe des Webservices.
        \item[\ref{q:four}]\hfill \\
            Da die Aufgabe darin bestand, Github-Repositories zu analysieren, konnte in den Docker-Container der \ac{API} git mitinstalliert werden, wodurch ein voller Zugriff der \ac{API} auf den Funktionsumfang von git besteht.
            Durch diesen erlangte die \ac{API} die Möglichkeit Repositories zu clonen und anschließend intern weiterzuverarbeiten, was den Funktionsumfang der selbst-implementierten Methoden aus den Punkten $I$ bis $III$ der Forschungsfragen bereits entsprang.
        \item[\ref{q:five}]\hfill \\
            Die Frage nach der Laufzeitverbesserung stellte sich nach der Umsetzung der V1 (näheres in \ref{sec:Implementation1}), da dort die Wartezeiten für die Abfrage von Paketen den zeitlichen Rahmen übertraf.
            Nach der Umsetzung einer Pipeline zur Verbesserung der Laufzeit überbot man jedoch ebenfalls die selbstgesetzt zeitliche Beschränkung der Suche eines Paketes an sich.
            \\
            Die Überwindung der Zeitmauer gelang dann durch die in der V2 (\ref{sec:Implementation2}) verwendeten anderen Datenbank -- dann MySQL.
    \end{description}
    Der hauptsächliche Auftrag des Projektes bestand in der paraphrasierten Frage, wie transitive Schwachstellen ermittelt und ausgegeben werden können.
    Dies wurde erfolgreich umgesetzt.
    \\
    Die möglichen Erweiterungen -- besprochen in der \ref{sec:Diskussion} \nameref{sec:Diskussion} -- sind aufbauend und benötigen keine konkreten Änderungen in dem Grundgerüst, welches mit dem vorliegendem Projekt erreicht wurde.
    \\ \\
    Der zu dem Zeitpunkt der Abgabe aktuelle commit im Github-Repository wie folgt getagt: \\
    \href{https://github.com/WSE-research/AmIVulnerable/releases/tag/v2.0}{github.com/WSE-research/AmIVulnerable/releases/tag/v2.0}
   

    % \section{Einleitung} \label{sec:Einleitung}
    \subsection{Problemstellung} \label{subsec:Problemstellung}
    Das Entwickeln von Softwarelösungen ist heutzutage nicht mehr ohne Bibliotheken, Frameworks oder externe Module denkbar.
    Solche Abhängigkeiten bestimmen Funktionalität, Effizienz und Sicherheit der jeweiligen Softwarelösung.
    Allerdings ist die Verwaltung dieser Abhängigkeiten mit ihren, teils transitiv vererbten, Sicherheitslücken mit steigender Zahl immer komplexer.
    Um nun also eine fundierte Entscheidung über eine Abhängigkeit treffen zu können muss die mangelnde Transparenz der Abhängigkeitsstruktur von Softwarelösungen berichtigt werden.
    \section{Motivation} \label{subsec:Motivation}
    Beim Entwickeln von Softwarelösungen gibt es viele Herausforderungen und Probleme. 
    Diese werden durch viele bereits vorhandene Softwarepakete, wie etwa Bibliotheken oder Frameworks, bewältigt.
    Die Nutzung frei verfügbarer Softwarepakete ist deshalb im Arbeitsalltag gang und gäbe.
    Freiwillige oder Hobby-Programmierer ermöglichen mit ihrem Einsatz, dass weltweit die Entwicklung neuer Software sowohl im kommerziellen als auch privaten und öffentlichen Bereich vereinfacht, vereinheitlicht und beschleunigt wird.
    Dank der Konkurrenz freier Pakete, zum Beispiel anhand ihrer Nutzungszahl, gestaltet sich dort ein Wettbewerb, der Pakete (a) an Bedeutung verlieren lässt bzw. (b) soweit verbessert, dass ihre Funktionen und Benutzbarkeit anschließend überzeugen konnten.

    Neben der Funktionalität besteht ein anderer essentieller Aspekt in der Sicherheit.
    Eben jene muss sich bei jedem Paket separat und gekapselt gesehen auf einem solchem Niveau befinden, dass ihre Verwendung keine fahrlässig Gefahr darstellt.
    Dies beginnt bei zu kurzen Schlüssellängen und endet bei komplexen Programmen mit verschiedenen Angriffsschwachstellen.
    Aber die Aufgabe, für jedes verwendete Paket einzeln die Sicherheitslücken nachzulesen oder für eine Paketsammlung nachzuvollziehen, ist selbst mit dem vorhandenen Angeboten zeitaufwendig und ressourcenintensiv -- schließlich werden so personelle Kräfte und Rechenkapazitäten gebunden.

    Daher besteht ein dringender Bedarf an einem Werkzeug, dass Entwicklern und Managern eine transparente und umfassende Analyse der Abhängigkeiten auf Sicherheitslücken ganzer Projekte bietet.
    Ein solches Automatisierungswerkzeug für die Analyse von Paketen trägt somit einer frühzeitigen Erkennung von Sicherheitslücken und -risiken bei und reduziert Zeitaufwand den der Nutzer.
    Damit kann die Softwarequalität verbessert werden, indem erkannt werden kann, ob eine genutzte Softwarekomponente aktualisiert oder ausgetauscht werden muss.
    Hieraus ergeben sich folgende Forschungsfragen (FF):
    \begin{enumerate}[label=\textbf{FF-\Roman*}, leftmargin=1.75cm]
        \item \textbf{Welche Funktionen sind notwendig um ein Paket auf Sicherheitslücken zu untersuchen?}\label{q:one}
        \item \textbf{Welche Funktionen sind notwendig um ein Repository auf Sicherheitslücken durch Abhängigkeiten zu untersuchen?}\label{q:two}
        \item \textbf{Wie sind die Resultat-Daten für den Endnutzer sowie Maschinen zu strukturieren, damit jene ohne weitere aufwändige Aufbereitung zu verarbeiten sind?}\label{q:three}
        \item \textbf{Wie ist ein Schwachstellen-Analyse-Tool in Betracht auf transitive Anhängig\-keiten von Repositories zu implementieren?}\label{q:four}
        \item \textbf{Wie kann ein solches Tool leistungsstärker sowie laufzeiteffizienter werden?}\label{q:five}
    \end{enumerate}
    Insgesamt bringt solch ein Werkzeug Unternehmen verschiedene Vorteile, wie in der Softwareentwicklung, darunter Transparenz, Qualität sowie Sicherheit.
    Die Gesamtsicherheit von Applikationen sowie auch Unternehmen kann durch nun sichtbare Schwachstellen in tiefen Ebenen der Softwareabhängigkeit verbessert werden.
    Dieses Werkzeug hilft also bei der Entscheidungsfindung über externe Softwarekomponenten sowie bei der Aufdeckung von Sicherheitsrisiken.
    
    \subsection{Vorgehen} \label{subsec:Vorgehen}
Um nun ein solches Werkzeug zur Verfügung zu stellen muss zuerst eine Anforderungsanalyse durchgeführt werden.
\begin{itemize}
    \item Transparenz schaffen
    \item Entscheidungsfindung unterstützen
    \item Sicherheitsrisiken minimieren
\end{itemize}
Diese Ziele sollen folgenden Nutzern bei der Entscheidungsfindung über Abhängigkeiten zu nutzen sein:
\begin{itemize}
    \item Softwareentwickler
    \item Projekt-Owner bzw. Teamleiter
\end{itemize}
Folgende Funktionalitäten sollen durch dieses Werkzeug umgesetzt werden:
Dazu sind folgende Ziele des Werkzeugs festgesetzt:
\begin{itemize}
    \item Information über Sicherheitslücken eines Pakets und dessen Abhängigkeiten.
    \item Information über Sicherheitslücken einer Liste von Paketen und dessen Abhängigkeiten.
    \item Analyse eines Git-Projekts mit dessen Abhängigkeiten auf Sicherheitslücken.
    \item Bereitstellung der Sicherheitslücken als weiter nutzbares Datenformat.
\end{itemize}
    % \section{Recherche} \label{sec:Recherche}
    \subsection{CVE} \label{subsec:CVE}
CVE, oder Common Vulnerabilities and Exposures \cite{}
    \subsection{Umgang mit CVE} \label{subsec:Umgang_mit_CVE}
Umgang mit CVE
    \subsection{Referenzen} \label{subsec:Referenzen}
Referenzen und CVE-Nutzungen
    \subsubsection{\ac{NIST}-\ac{API}} \label{subsubsec:NIST_API}
    \ac{NIST}-\ac{API}
    \subsubsection{DEPENDA-BOT} \label{subsubsec:DEPENDA_BOT}
    DEPENDA-BOT (GITHUB)
    \subsubsection{Weitere} \label{subsubsec:Weitere}
    Weitere
    % \section{Umsetzung} \label{sec:Umsetzung}
    \subsection{Technologiestack} \label{subsec:Technologiestack}
Technologiestack
    ASP.NET
    DOCKER
    LiteDB (NOSQL) erst
    MySQL (SQL) Ausblick
    \subsection{\ac{API}} \label{subsec:API}
    \subsubsection{Modells} \label{subsubsec:Modells}
PUML (Klassen)
Interaktion und Zusammenhang der Klassen/Fktn
    \subsubsection{Controller} \label{subsubsec:Controller}
Auflistung und Erklärung (Swagger screenshot) in Paragraphs
Views (JSON-LD def von CVE-Result, JSON-LD von NodePackageResult)
    \subsubsection{DB-Anbindung} \label{subsubsec:DB-Anbindung}
DB-Anbindung: LiteDB braucht DB-Controller
        
    % \section{Analyse} \label{sec:Analyse}
    \subsection{Laufzeitmessungen nach Datenbankerstellung} \label{subsec:Laufzeitmessungen}
Laufzeitmessungen nach Datenbankerstellung (etwa 1 std rechnen)
JSON-Datei Suche VS LiteDB suche (60 min vs 8 s): Rechnung nach wie vielen Suchanfragen sich das ganze gelohnt hat (nach etwa einer)
    \subsection{Laufzeitanalyse Mono-Suche und Pipeline-Suche auf LiteDB} \label{subsec:Laufzeitanalyse}
Laufzeitanalyse Mono-Suche und Pipeline-Suche auf LiteDB
Wie ist LiteDB-Suche zu verschnellern
Vergleich der Messungen
Erklärung der Pipeline (Diagramme) (Wie und Warum, LiteDB ist File-Basier und deswegen kann auf mehreren Dateien gesucht werden falls genug threads zur verfügung stehen)
    \subsection{Ausblick} \label{subsec:Ausblick}
Ausblick
Ziel: schneller als \ac{NIST}-\ac{API} sein (so und so viele Sekunden übers Netz im Vergleich zu so und so vielen sekunden über unsere \ac{API})
Um Ziel zu erreichen: Andere Datenbank verwenden um Lesezugriffe optimieren

    % \section{Conclusion} \label{sec:Conclusion}
    (5)
    (sub)
    Vision
        Validieren ob Ziel erreicht ist, dass
            ein Node-JS-Projekt komplett analysisert (alle Pakete entnommen werden) und nach Schwachstellen untersucht werden
            das zurückgegebene Format der API sinnvoll und struckturiert ist
            Wurde sich an programmiertechnische Standards gehalten
                Models sind eigene Bibliothek (losgelöst von der API, können weiterverwendet werden getrennt und weiterentwickelt werden)
                Controller liegen im Controller-Namespace
                Sinnvolle Trennung von Endpunkten in Controller mit http-Signatur
                Routennamen der Endpunkte in natürlicher Sprache/Pfaden
        Nächste Schritte mit neuer Zielsetzung
            API mit MySQL-DB versehen für vermutete höhere Performance (vermutung weil db-Server wahrscheinlich schneller als lokal lesen/schreiben[hardwareabhängig])
            Analyseergebnis ist auszubauen um Schweregrad der Schwachstelle um besonders kritische stellen im Dependency-Baum hervorzuheben
            

    \newpage
    \phantomsection
    \printbibheading[title={Quellen}]
    \addcontentsline{toc}{section}{Quellen}
    \printbibliography[type=book, heading=subbibliography, title={Literatur}]
    % \newpage
    % \printbibliography[type=online, heading=subbibliography, title={Links}]
    % \newpage
    % \printbibliography[type=article, heading=subbibliography, title={Zeitschriften}]

    \newpage
    \phantomsection
    \addcontentsline{toc}{section}{Abbildungen}
    \listoffigures
    \newpage
    \phantomsection
    \addcontentsline{toc}{section}{Tabellen}
    \listoftables

\end{document}
