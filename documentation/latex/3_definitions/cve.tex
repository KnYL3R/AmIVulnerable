\subsection{CVE} \label{sec:CVE}
Der Aufgabe Einschätzung der Sicherheit und Einhaltung von Standards hat sich die Mitre Corporation gestellt; eine us-amerikanische Forschungsabteilung der "National Cybersecurity FFRDC", die staatliche Finanzierung genießt.
CVE nennt sich ihr Referenziersystem und stellt dabei die englische Abkürzung \glqq Common Vulnerabilities and Exposures\grqq~dar.
\\
\glqq Common Vulnerabilities and Exposures\grqq~bzw. häufige Schwachstellen und Risiken sind als Liste öffentlich verfügbar.
Hierbei werden nur die Schwachstellen betrachtet, die durch eine \glqq CVE Numbering Authority\grqq (CNA) eine CVE-Nummer zugewiesen bekommen haben.\cite{}
Solch eine CVE-Nummer beinhaltet keine technischen Informationen zur Betroffenen Softwarekomponente.
Diese Informationen müssen über andere Services oder Datenbanken, wie z.B. in der \glqq U.S. National Vulnerability Database\grqq~oder der \glqq CERT/CC Vulnerability Notes Database\grqq.