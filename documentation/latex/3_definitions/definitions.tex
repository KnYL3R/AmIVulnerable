\section{Definitionen} \label{sec:Definitionen}
\begin{itemize}
    \item \textbf{CVE (Common Vulnerabilities and Exposures):} \\
    Der Aufgabe Einschätzung der Sicherheit und Einhaltung von Standards hat sich die Mitre Corporation gestellt; eine us-amerikanische Forschungsabteilung der "National Cybersecurity FFRDC", die staatliche Finanzierung genießt.
    CVE nennt sich ihr Referenziersystem und stellt dabei die englische Abkürzung \glqq Common Vulnerabilities and Exposures\grqq~dar.
    \\
    \glqq Common Vulnerabilities and Exposures\grqq~bzw. häufige Schwachstellen und Risiken sind als Liste öffentlich verfügbar.
    In dieser Liste werden nur die Schwachstellen betrachtet, die durch eine der 364 \glqq CVE Numbering Authority's\grqq (CNA's) eine CVE-Nummer zugewiesen bekommen haben.\cite{}
    Eine CVE-Nummer beinhaltet wiederum keine technischen Informationen zur Betroffenen Software- oder Hardwarekomponente sondern Produktname, Version, eine Beschreibung der Schwachstelle und gegebenenfalls werden Hinweise zur Behebung vermerkt.
    Diese Informationen müssen über andere Services oder Datenbanken, wie z.B. in der \glqq U.S. National Vulnerability Database\grqq~oder der \glqq CERT/CC Vulnerability Notes Database\grqq.

    \item \textbf{Direkte Abhängigkeiten:} \\
    In der Softwareentwicklung ist eine Abhängigkeit ein Softwarepaket, welches von der Anwendung selbst benötigt wird, um korrekt zu funktionieren.
    Damit sind direkte Abhängigkeiten nicht wie indirekte durch andere Abhängigkeiten eingeführt, sondern direkt eingebunden.
    Direkte Abhängigkeiten sind typischerweise externe Softwarekomponenten oder Bibliotheken.
    Diese werden auch als Paket bezeichnet. 

    \item \textbf{Transitive Abhängigkeiten:} \\
    Transitive Abhängigkeiten sind indirekte Abhängigkeiten, die durch andere Abhängigkeiten benötigt werden.
    Wenn nun eine Applikation ein Paket einbindet welches ein weiteres Paket selbst benötigt, so ist dieses eine transitive Abhängigkeit der Applikation.
    Somit können sich rekursiv Abhängigkeitsbäume aufspannen, die teils sogar gleiche Pakete in verschiedenen Versionen einbinden.
    
    \item \textbf{API (Application Programming Interface):} \\
    Eine API definiert eine Reihe von Regeln und Mechanismen, über die verschiedene Softwarekomponenten miteinander interagieren können.
    Somit bieten stellen sie Schnittstelle zwischen verschiedenen Systemen dar.
    Sie legt fest, wie Softwaremodule oder -anwendungen miteinander kommunizieren, indem sie Funktionen, Methoden, Protokolle und Datenstrukturen bereitstellt.
    Um eine API zu nutzen müssen diese Festlegungen eingehalten werden.
    
    
    \item \textbf{JSON (JavaScript Object Notation):} \\
    JSON ist ein in der Webentwicklung verbreitetes Datenformat.
    Genutzt wird dieses um strukturierte Daten zu speichern und auszutauschen.
    Basierend auf einer einfachen Syntax, die auf Schlüssel-Wert-Paaren und verschachtelten Datenstrukturen basiert, ermöglicht dieses Format für Mensch sowie Maschine eine gute Les- und Parsbarkeit.
    JSON wird häufig für die Übertragung von Daten zwischen Client und Server.
    Dabei ist es programmiersprachenunabhängig verwendet allerdings Konventionen ähnlich der von C, C++ oder C\#.

    \item \textbf{JSON-LD (JSON-Linked Data):} \\
    JSON-LD ist ein Linked-Data-Format, welches auf dem JSON-Format basiert.
    Dieses Linked-Data-Format ermöglicht es ein standardisiertes, maschinenlesbares Datennetzwerk über Websites hinweg zu schaffen.
\end{itemize}