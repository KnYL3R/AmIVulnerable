\subsection{Motivation} \label{subsec:Motivation}
    Beim Entwickeln von Softwarelösungen gibt es viele Herausforderungen und Probleme. 
    Diese werden durch viele bereits vorhandene Softwarepakete bewältigt.
    Die Nutzung frei verfügbarer Softwarepakete sind deshalb im Arbeitsalltag gang und gäbe.
    Freiwillige oder Hobby-Programmierer ermöglichen mit ihrem Einsatz, dass weltweit die Entwicklung neuer Software sowohl im kommerziellen als auch privaten und öffentlichen Bereich vereinfacht, vereinheitlicht und beschleunigt wird.
    Dank der Konkurrenz freier Pakete, zum Beispiel anhand ihrer Nutzungszahl, gestaltet sich dort ein Wettbewerb, der gute Pakete beständig besser werden lässt und nicht durchdachte entweder (a) an Bedeutung verlieren lässt oder (b) soweit verbessert, dass ihre Funktionen und Benutzbarkeit anschließend überzeugen konnten.
    Ein anderer essentieller Aspekt außer der Nutzbarkeit oder Funktionserfüllung ist die Sicherheit.
    Eben jene muss sich bei jedem Paket separat und gekapselt gesehen auf einem solchem Niveau befinden, dass ihre Verwendung keine fahrlässig Gefahr darstellt.
    Dies beginnt bei zu kurzen Schlüssellängen und endet bei komplexen Programmen mit verschiedenen Angriffsschwachstellen.
    \\ \\
    Der Aufgabe Einschätzung der Sicherheit und Einhaltung von Standards hat sich die Mitre Corporation gestellt; eine us-amerikanische Forschungsabteilung der "National Cybersecurity FFRDC", die staatliche Finanzierung genießt.
    CVE nennt sich ihr Referenziersystem und stellt dabei die englische Abkürzung \glqq Common Vulnerabilities and Exposures\grqq~dar.
    \\
    Aber die Aufgabe, für jedes verwendete Paket einzeln die Sicherheitslücken nachzulesen oder für eine Paketsammlung nachzuvollziehen, ist selbst mit dem Angebot der \glqq National Cybersecurity FFRDC\grqq~zeitaufwendig und ressourcenintensiv - schließlich werden so personelle Kräfte und Rechenkapazitäten gebunden.
    Eine Automatisierung der Analyse solcher Pakete zielt somit nicht nur eine Reduktion des Zeitaufwandes mit sich, auch ist eine umfangreichere Analyse ohne Mehraufwand möglich.
    Dies spiegelt sich beispielsweise in der Möglichkeit wieder, ganze Projekte direkt analysieren zu lassen anstelle der einzelnen Pakete.
    \\
    Das zeigt, dass ein dringender Bedarf an einem Werkzeug besteht, das Entwicklern und Managern eine transparente und umfassende Analyse der Abhängigkeiten auf Sicherheitslücken bietet.
    Solch ein Werkzeug könnte einem Unternehmen weiterhin verschiedene Vorteile im Bereich der Softwareentwicklung für Transparenz, Qualität und Sicherheit bringen.
    \begin{enumerate}
        \item Transparenz und Vertrauen\\
            Durch eine klare Übersicht über genutzte externe Softwarekomponenten mit Sicherheitslücken können frühzeitig besser informierte Entscheidungen getroffen und potenzielle Risiken identifiziert werden.
        \item Qualität\\
            Entwickler können leichter erkennen, ob Abhängigkeiten ersetzt oder aktualisiert werden müssen.
        \item Sicherheit\\
            Um die Gesamtsicherheit einer Applikation bestmöglich zu gewährleisten trägt ein solches Tool, welches Schwachstellen und Abhängigkeiten auch in tieferen Ebenen von Abhängigkeiten darstellt, zur Identifikation und Vermeidung von Sicherheitslücken bei. 
    \end{enumerate}
    Insgesamt hilft ein solches Werkzeug bei der Entscheidungsfindung über externe Softwarekomponenten sowie bei der Aufdeckung von Sicherheitsrisiken.