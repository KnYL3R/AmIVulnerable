\subsection{Vorgehen} \label{subsec:Vorgehen}
Um nun ein solches Werkzeug zur Verfügung zu stellen muss zuerst eine Anforderungsanalyse durchgeführt werden.
\begin{itemize}
    \item Transparenz schaffen
    \item Entscheidungsfindung unterstützen
    \item Sicherheitsrisiken minimieren
\end{itemize}
Diese Ziele sollen folgenden Nutzern bei der Entscheidungsfindung über Abhängigkeiten zu nutzen sein:
\begin{itemize}
    \item Softwareentwickler
    \item Projekt-Owner bzw. Teamleiter
\end{itemize}
Folgende Funktionalitäten sollen durch dieses Werkzeug umgesetzt werden:
Dazu sind folgende Ziele des Werkzeugs festgesetzt:
\begin{itemize}
    \item Information über Sicherheitslücken eines Pakets und dessen Abhängigkeiten.
    \item Information über Sicherheitslücken einer Liste von Paketen und dessen Abhängigkeiten.
    \item Analyse eines Git-Projekts mit dessen Abhängigkeiten auf Sicherheitslücken.
    \item Bereitstellung der Sicherheitslücken als weiter nutzbares Datenformat.
\end{itemize}