\section{Zusammenfassung} \label{sec:Zusammenfassung}
    Die in der \nameref{subsec:Motivation} aufgelisteten Forschungsfragen können wie folgt nach Bearbeitung des Auftrages beantwortet werden:
    \begin{description}
        \item[\ref{q:one}]\hfill \\
            Um die Pakete zu untersuchen ist es notwendig, die \ac{CVE}-Daten in eine interne Datenbank umzuwandeln, da das Suchen auf den Rohdaten -- den \ac{CVE}-\ac{JSON}-Dateien -- zu zeitaufwendig ist.
            Dazu wurde schlussendlich eine MySQL-Datenbank ausgewählt, die dank der Indizierung einer Spalte die gewünschten Leistungsmerkmale aufweist.
        \item[\ref{q:two}]\hfill \\
            In der umgesetzten Version ist es dank der nativen Unterstützung von npm gelungen, die Abhängigkeiten der Pakte als \ac{JSON} zu extrahieren und dann intern so weiterzuverarbeiten, dass eine logische Repräsentation erfolgen kann.
        \item[\ref{q:three}]\hfill \\
            Die Lösung dieser Frage geschah über die einheitliche Verwendung von \ac{JSON} respektive \ac{JSON-LD} bei der Rückgabe des Webservices.
        \item[\ref{q:four}]\hfill \\
            Da die Aufgabe darin bestand, Github-Repositories zu analysieren, konnte in den Docker-Container der \ac{API} git mitinstalliert werden, wodurch ein voller Zugriff der \ac{API} auf den Funktionsumfang von git besteht.
            Durch diesen erlangte die \ac{API} die Möglichkeit Repositories zu clonen und anschließend intern weiterzuverarbeiten, was den Funktionsumfang der selbst-implementierten Methoden aus den Punkten $I$ bis $III$ der Forschungsfragen bereits entsprang.
        \item[\ref{q:five}]\hfill \\
            Die Frage nach der Laufzeitverbesserung stellte sich nach der Umsetzung der V1 (näheres in \ref{sec:Implementation1}), da dort die Wartezeiten für die Abfrage von Paketen den zeitlichen Rahmen übertraf.
            Nach der Umsetzung einer Pipeline zur Verbesserung der Laufzeit überbot man jedoch ebenfalls die selbstgesetzt zeitliche Beschränkung der Suche eines Paketes an sich.
            \\
            Die Überwindung der Zeitmauer gelang dann durch die in der V2 (\ref{sec:Implementation2}) verwendeten anderen Datenbank -- dann MySQL.
    \end{description}
    Der hauptsächliche Auftrag des Projektes bestand in der paraphrasierten Frage, wie transitive Schwachstellen ermittelt und ausgegeben werden können.
    Dies wurde erfolgreich umgesetzt.
    \\
    Die möglichen Erweiterungen -- besprochen in der \ref{sec:Diskussion} \nameref{sec:Diskussion} -- sind aufbauend und benötigen keine konkreten Änderungen in dem Grundgerüst, welches mit dem vorliegendem Projekt erreicht wurde.
    \\ \\
    Der zu dem Zeitpunkt der Abgabe aktuelle commit im Github-Repository wie folgt getagt: \\
    \href{https://github.com/WSE-research/AmIVulnerable/releases/tag/v2.0}{github.com/WSE-research/AmIVulnerable/releases/tag/v2.0}
