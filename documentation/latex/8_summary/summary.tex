\section{Zusammenfassung} \label{sec:Zusammenfassung}
    Aus der \nameref{subsec:Motivation} kann der hauptsächliche Auftrag mit der Forschungsfragen \ref{q:four} umformuliert werden mit der Frage, wie transitive Schwachstellen ermittelt und ausgegeben werden können.
    \\
    Die Bearbeitung kann unter dem Abschnitt \ref{sec:Implementation} \nameref{sec:Implementation} nachvollzogen werden.
    Nach anfänglicher Umsetzung, die die grundlegenden Funktionen vorweisen konnte, musste jedoch festgestellt werden, dass die nicht-funktionalen Anforderungen mit der Wahl einer dateibasierten Datenbank zügig implementierbar, aber nicht erreichbar sind.
    Somit erfolgte eine Version 2, die mit der Änderung der Datenbank zu einer MySQL-Datenbank, wodurch die Ziele -- insbesondere die der zeitlichen Vorgaben -- zufriedenstellend erreicht werden konnte.
    \\
    Während die \ac{API} anfänglich ein hauptaugenmerk auf die Funktion der Analyse einzelner Pakete fokussierte um diese Funktionalität zu gewährleisten, ist als zweite Iteration die Aufgabe der transitiven Analyse angegangen worden.
    Da über beispielsweise die NIST-API (\ref{sec:NIST-API}) eine einzelne Paketanalyse ebenfalls machbar ist, lag hier der Schwerpunkt des Auftrages.
    \\
    Nach der erfolgreichen Umsetzung der Baumstrukturaufstellung in einem npm-Projekt musste anschließend lediglich die Verbindung der Paketanalyse und der Aufbereitung der Ergebnisse als \textit{Dep\-en\-dencies-Tree} erfolgen.
    Die Ausgabe jenes Ergebnisses geschieht dann im \ac{JSON-LD} Format, damit jeder Identifikator eindeutig und weiterverarbeitbar ist.
    \\
    Insgesamt wurden die Anforderungen aus der \nameref{subsec:Motivation} also erfüllt, können jedoch noch weiter ausgearbeitet werden durch zum Beispiel die Erweiterung um weitere Frameworks als das aktuell einzige npm.
    % TODO: was noch? 
    % TODO: hab ich was ausgelassen?
    % TODO: war das zu strukturiert und eigentlich gar nicht vorher beschrieben mit der Historie?
