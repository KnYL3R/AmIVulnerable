\section{Motivation} \label{subsec:Motivation}
    Beim Entwickeln von Softwarelösungen gibt es viele Herausforderungen und Probleme. 
    Diese werden durch viele bereits vorhandene Softwarepakete, wie etwa Bibliotheken oder Frameworks, bewältigt.
    Die Nutzung frei verfügbarer Softwarepakete ist deshalb im Arbeitsalltag gang und gäbe.
    Freiwillige oder Hobby-Programmierer ermöglichen mit ihrem Einsatz, dass weltweit die Entwicklung neuer Software sowohl im kommerziellen als auch privaten und öffentlichen Bereich vereinfacht, vereinheitlicht und beschleunigt wird.
    Dank der Konkurrenz freier Pakete, zum Beispiel anhand ihrer Nutzungszahl, gestaltet sich dort ein Wettbewerb, der Pakete (a) an Bedeutung verlieren lässt bzw. (b) soweit verbessert, dass ihre Funktionen und Benutzbarkeit anschließend überzeugen konnten.
    \\ \\
    Neben der Funktionalität besteht ein anderer essentieller Aspekt in der Sicherheit.
    Eben jene muss sich bei jedem Paket separat und gekapselt gesehen auf einem solchem Niveau befinden, dass ihre Verwendung keine fahrlässig Gefahr darstellt.
    Dies beginnt bei zu kurzen Schlüssellängen und endet bei komplexen Programmen mit verschiedenen Angriffsschwachstellen.
    Aber die Aufgabe, für jedes verwendete Paket einzeln die Sicherheitslücken nachzulesen oder für eine Paketsammlung nachzuvollziehen, ist selbst mit dem vorhandenen Angeboten zeitaufwendig und ressourcenintensiv -- schließlich werden so personelle Kräfte und Rechenkapazitäten gebunden.
    \\ \\
    Daher besteht ein dringender Bedarf an einem Werkzeug, dass Entwicklern und Managern eine transparente und umfassende Analyse der Abhängigkeiten auf Sicherheitslücken bietet.
    Ein solches Automatisierungswerkzeug für die Analyse von Paketen trägt somit einer frühzeitigen Erkennung von Sicherheitslücken und -risiken bei und reduziert Zeitaufwandt den der Nutzer.
    Damit kann die Softwarequalität verbessert werden, indem erkannt werden kann, ob eine genutzte Softwarekomponente aktualisiert oder ausgetauscht werden muss.
    \\ \\
    Insgesamt bringt solch ein Werkzeug Unternehmen verschiedene Vorteile, wie in der Softwareentwicklung, darunter Transparenz, Qualität sowie Sicherheit.
    Die Gesamtsicherheit von Applikationen sowie auch Unternehmen kann durch nun sichtbare Schwachstellen in tiefen Ebenen der Softwareabhängigkeit verbessert werden.
    Dieses Werkzeug hilft also bei der Entscheidungsfindung über externe Softwarekomponenten sowie bei der Aufdeckung von Sicherheitsrisiken.