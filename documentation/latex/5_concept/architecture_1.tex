\subsubsection{Architektur V1} \label{sec:Architektur}
    Für die Beantwortung der Forschungsfragen \ref{q:one}, \ref{q:two}, \ref{q:five} und vor allem \ref{q:four} verschiedene Komponenten nötig:
    \begin{enumerate}
        \item \textbf{Framework} \label{arch_1}\\
            Um eine \ac{API} zu entwickeln, muss ein passendes Framework genutzt werden.
            Dadurch, dass keine Notwendigkeit für ein Frontend besteht gibt es hier wenig Eingrenzungen.
            \\
            Aus Erfahrungsgründen wurde das C\# Framework ASP.NET gewählt.
            Mit diesem lassen sich unter anderem \ac{API}-Services bauen.
            Mittels der für das .NET Framework explizit vorhandenen IDE \glqq Visual Studio\grqq~bestehen auch native Debugging-Möglichkeiten.
            Durch die große und aktive Community von ASP.NET ist dieses Framework sehr gut dokumentiert und es existieren viele Open-Source-Bibliotheken.
        \item \textbf{Datenbank} \label{arch_2}\\
            In der Datenbank sind alle \ac{CVE}-Daten zu persistieren, welche zu durchsuchen sind.
            Diese Daten dienen als Grundlage der Identifizierung von Schachstellen in Paketen innerhalb von Projekten.
            Wichtig sind hier schnelle Lesezugriffe, da diese beim Abfragen der Datenbank die größte Laufzeiteinsparung bringen.
            \\ \\
            Durch die Wahl des Frameworks auf ASP.NET ist es möglich einen \textit{embedded NoSQL-Document-Store} -- LiteDB zu nutzen. % TODO: https://www.litedb.org/
            Diese ist leicht intern nutzbar und muss als \textit{file-based DB} nicht separat explizit gestartet und verwaltet werden, womit diese auch im Container der \ac{API} mit enthalten ist.
        \item \textbf{Controller} \label{arch_3}\\
            Controller einer \ac{API} nehmen HTTP-Anfragen entgegen und reagieren darauf.
            Hier wird die Hauptaufgabe der \ac{API} geschehen, da alle Funktionalitäten, sei es Datenbankabfragen, Klonen eines zu untersuchenden Repositories oder die Untersuchung dieses, in einem solchen implementiert oder aufgerufen werden müssen.
            \\ \\
            Notwendig sind hier vier Controller.
            (1) Für die Forschungsfrage \ref{q:two} bzw. \ref{q:four} muss ein Git-Controller zum nutzen von \ac{CVE}-Daten sowie zum Erhalt von zu analysierenden Repositories entstehen.
            In diesem sind Endpunkte zum clonen des \ac{CVE}-Daten-Repositories sowie zum clonen des Analyse-Repositories zu implementieren. % TODO: clone CVE-DATEN-REPO-LINK
            \\
            Weiterhin ist (2) ein Controller für Abhängigkeiten nötig, in dem man aus dem zu analysierenden Repositoriy den Abhängigkeitsbaum extrahiert sowie diesen mit Schwachstellendaten anreichert.
            \\
            Für die Untersuchung einzelner Pakete und Listen dieser ist ein weiterer (3) Endpunkt zu implementieren.
            In diesem ist auch die Update-Funktion der Datenbasis hinzuzufügen.
            \\
            Für Forschungsfrage \ref{q:three} muss in jedem Endpunkt (4) bei korrekter Antwort ein Context mitgeliefert werden, damit der gelieferte Inhalt so durch JSON-LD zu interpretieren ist.
            Ebenfalls sind durch einen Controller die Rückgabedaten zu dokumentieren.
            Dazu ist zwischen Softwarepaketen und \ac{CVE}-Einträgen zu unterscheiden.
        \item \textbf{Datenmodelle} \label{arch_4}\\
            Um Daten korrekt in die Datenbank einzufügen, um ein Resultat-JSON zu erzeugen oder die Paketliste intern zu verarbeiten -- dazu sind Datenmodelle nötig.
        \item \textbf{Konvertierung von und in JSON} \label{arch_5}\\
            Beim Einlesen der \ac{CVE}-Daten in die Datenbank ist eine Konvertierung vom vorhandenen JSON-Format in Einträge der Datenbank vorzunehmen.
            Die aus der Datenbank genutzten, durch den Controller verarbeiteten, Daten müssen nun schließlich im JSON-Format dem Benutzer übermittelt werden.
            \\
            Dies muss in den jeweiligen Controllern geschehen.
            Damit die Daten besser weiterverwendbar sind muss zusätzlich ein Kontext \glqq @context\grqq~hinzugefügt werden.
        \item \textbf{Container} \\
            Um die \ac{API} unabhängig von der Umgebung und möglichst speicheraufwandsschmal zu nutzen muss die Anwendung \glqq Containerisiert\grqq~werden.
            \\ \\
            Für den Bau der Containers wird Docker-Compose genutzt.
    \end{enumerate}
    Weiterhin muss das JSON-LD-Format für die Rückgabedaten definiert werden (Forschungsfrage \ref{three}).
