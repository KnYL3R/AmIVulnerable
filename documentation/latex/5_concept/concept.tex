\section{Konzept} \label{sec:Konzept}
Im Rahmen dieser Arbeit soll eine API zur Verfügung gestellt werden, die es ermöglicht transitive Vererbungen von Abhängigkeiten darzustellen.
Dabei sollen einzelne Pakete, Listen von Paketen und ganze Projekte selbst analysiert werden können.
Abhängigkeiten werden durch Extraktion eines Abhängigkeitsbaumes erkannt und einzeln gegen Schwachstellendaten geprüft.
Der Nutzer, bzw. Maschinen sollen letztendlich ein weiterverwendbares klares Datenformat erhalten in welchem betroffene Abhängigkeiten und Ihre Schwachstellen vermerkt sind.
\\ \\
Folgende Forschungsfragen werden dabei betrachtet:
\begin{enumerate}
    \item \textbf{Welche Funktionen sind notwendig um ein Projekt auf Sicherheitslücken durch Abhängigkeiten zu untersuchen?} \label{one}
    \item \textbf{Wie sind die Resultatdaten für den Endnutzer sowie Maschinen besser weiterzuverwenden?} \label{two}
    \item \textbf{Wie ist eine Schwachstellenanalyse-API in Betracht auf transitive Anhängigkeiten von Repositories zu implementieren?} \label{three}
    \item \textbf{Wie kann eine solche API leistungsstärker und laufzeiteffizienter werden?} \label{four}
\end{enumerate}
Forschungsfrage \ref{one} bestimmt die zu implementierenden Funktionalitäten.
Dazu muss zuerst betrachtet werden, welche Ausgangsdaten vorliegen.
Die zu analysierenden Repositories bzw. Abhängigkeiten und die genutzten Schwachstellendaten.
Je nach Größe der zu suchenden Menge an Abhängigkeiten müssen hier verschiedene Algorithmen genutzt werden um eine effiziente Ergebnisfindung zu gestalten.
Hier ist auch die Suche der Abhängigkeiten auf der Datenbasis zu thematisieren. 
\\ \\
Forschungsfrage \ref{two} definiert den Rückgabetyp der Daten.
Hier müssen sinnvolle Rückgabedaten für den Endnutzer identifiziert sowie eine Definition dieser vorgenommen werden.
\\ \\
Forschungsfrage \ref{three} behandelt die Umsetung der API selbst.
Hierfür wird eine Programmiersprache mit zugehörigem Framework ausgewählt sowie eine Swagger zur Nutzung und Dokumentation der Endpunkte eingesetzt.
Es muss sich in diesem Umfeld auch auf eine Programmiersprache im Sinne des zu Untersuchenden Projekte entschieden werden um den Umfang dieser Arbeit möglichst genau zu halten.
Der entstandene Service muss weiterhin im Container-Kontext nutzbar sein.
\\ \\
Forschungsfrage \ref{four} beschäftigt sich mit der Optimierung der API.
Hier müssen algorithmische sowie technologische Verbesserungen in Leistung oder Laufzeit aufgezeigt und mögliche Lösungswege aufgezeigt oder implementiert werden.
Besonders gewählte Datenbankformate werden hier thematisiert.