\subsection{Besondere Merkmale} \label{sec:Besondere Merkmale}
    Dadurch, dass diese \ac{API} bei schwachstellenbefallenen Abhängigkeiten den Abhängigkeitsbaum darstellt, ist für den Nutzer eine hohe Granularität und Transparenz der Abhängigkeit gegeben.
    Somit können Aufwand, bezüglich Ersatzes oder Anpassungen bei Abhängigkeiten, besser eingeschätzt werden und die Entscheidung, hier bezüglich der Dringlichkeit dieses Schrittes des Ersatzes, zutreffender fallen.
    \\ \\
    Durch die Menge der als Schwachstellen-Suchbasis dienenden \ac{CVE}-Daten und deren stetige Erweiterung und Aktualisierung ist es der \ac{API} zu ermöglichen, immer nur die aktuellsten Änderungen herunterzuladen und einzupflegen.
    Damit ist es nicht notwendig bei jeglichen Aktualisierungen die gesamte Datenbasis erneut herunterzuladen.
    \\ \\
    Durch die lokal vorliegende Datenbasis ist das Suchen deutlich schneller als diese Daten immer erneut extern anzufragen.
    Somit kann mit einer niedrigeren Laufzeit in z.B. CI/CD Intergrationen dieser gerechnet werden als andere Tools.
    \\ \\
    Die Abfrage einzelner Pakete ist ebenfalls unterstützt.
    Demnach ist die punktuelle Analyse einzelner Pakete möglich, was den Einsatzrahmen für die Endnutzer erweitert.
