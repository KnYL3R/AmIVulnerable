\subsection{Forschungsfragen} \label{sec:Forschungsfragen}
    Folgende Forschungsfragen (FF) werden in dieser Arbeit thematisiert:

    \ref{q:one}:
    Hier muss eine Betrachtung der notwendigen Funktionalitäten zur Analyse von einzelnen oder Listen von Abhängigkeiten auf Basis der eingeladenen Schwachstellendaten stattfinden.
    Je nachdem, ob eine Menge an Abhängigkeiten oder eine Einzelne gesucht werden muss müssen hier verschiedene Algorithmen implementiert werden um eine effiziente Ergebnisfindung zu gestalten.

    \ref{q:two}:
    Hier handelt es sich um die zu implementierenden Funktionalitäten für Repositories und geschieht aufbauend auf der Forschungsfrage \ref{q:one}.
    Dazu muss zuerst betrachtet werden, welche Ausgangsdaten vorliegen.
    Dazu gehören die zu analysierenden Repositories bzw. Abhängigkeiten und die genutzten Schwachstellendaten.

    \ref{q:three}:
    Hier wird der Rückgabetyp der Daten definiert.
    Es müssen sinnvolle Rückgabedaten für den Endnutzer identifiziert sowie eine Definition dieser vorgenommen werden.

    \ref{q:four}:
    Diese Forschungsfrage behandelt die Umsetzung der \ac{API} selbst.
    Es ist die Umsetzung der zu definierenden Funktionalen Anforderungen unter der Betrachtung der nicht-funktionalen Anforderungen.
    Hierfür wird eine Programmiersprache mit zugehörigem Framework ausgewählt sowie das End\-punkt\-doku\-mentations- und Test\-möglich\-keits\-tool.
    Es muss sich in diesem Umfeld auch auf eine Programmiersprache im Sinne der zu untersuchenden Projekte entschieden werden um den Umfang dieser Arbeit möglichst genau zu halten.
    Der entstandene Service muss weiterhin im Container-Kontext nutzbar sein.

    \ref{q:five}:
    Hier wird die Optimierung der \ac{API} betrachtet.
    Es müssen algorithmische sowie technologische Verbesserungen in Leistung oder Laufzeit aufgezeigt sowie deren mögliche Lösungswege implementiert oder betrachtet werden.
    Ein besonders Augenmerk liegt hier auf die gewählten Datenbankformate.
