\subsection{Forschungsfragen} \label{sec:Forschungsfragen}
    Aus den funktionalen und nichtfunktionalen Anforderungen ergeben sich folgende Forschungsfragen für diese Arbeit:
    \\ \\
    Forschungsfrage \ref{q:one}, welche aus den funktionalen Anforderungen \ref{f:one} und \ref{f:two} hervorkommt, betrachtet die Analyse von einzelnen oder Listen von Abhängigkeiten auf Basis der eingeladenen Schwachstellendaten.
    Je nachdem, ob eine Menge an Abhängigkeiten oder eine Einzelne gesucht werden muss müssen hier verschiedene Algorithmen implementiert werden um eine effiziente Ergebnisfindung zu gestalten.
    \\ \\
    Forschungsfrage \ref{q:two} handelt von den zu implementierenden Funktionalitäten für Repositories und geschieht aufbauen auf Forschungsfrage \ref{q:one}.
    Deshalb sind die folgenden funktionalen Anforderungen \ref{f:one}, \ref{f:three}, \ref{f:four} und \ref{f:five} zu betrachten.
    Dazu muss zuerst betrachtet werden, welche Ausgangsdaten vorliegen.
    Zusätzlich auch die zu analysierenden Repositories bzw. Abhängigkeiten und die genutzten Schwachstellendaten.
    \\ \\
    Forschungsfrage \ref{q:three} definiert den Rückgabetyp der Daten.
    Vor allem die nichtfunktionale Anforderung \ref{nf:four}, der Rückgabe im JSON-LD-Format, ist Ausschlaggebend.
    Hier müssen demnach sinnvolle Rückgabedaten für den Endnutzer identifiziert sowie eine Definition dieser vorgenommen werden.
    \\ \\
    Forschungsfrage \ref{q:four} behandelt die Umsetung der API selbst.
    Es ist die Umsetzung der Funktionalen Anforderungen unter der Betrachtung der nicht funktionalen Anforderungen.
    Vor allem nichtfunktionale Anforderung \ref{nf:five}, die Nutzung von Etablierten Technologien, hat hier ihren Schwerpunkt.
    Hierfür wird eine Programmiersprache mit zugehörigem Framework ausgewählt sowie das Endpunktdokumentations- und Testmöglichkeitstool.
    Es muss sich in diesem Umfeld auch auf eine Programmiersprache im Sinne der zu untersuchenden Projekte entschieden werden um den Umfang dieser Arbeit möglichst genau zu halten.
    Der entstandene Service muss, wie durch funktionale Anforderung \ref{f:seven} bestimmt, weiterhin im Container-Kontext nutzbar sein.
    \\ \\
    Forschungsfrage \ref{q:five} beschäftigt sich mit der Optimierung der API.
    Funktionale Anforderung \ref{f:six} und nichtfunktionale Anforderungen \ref{nf:one}, \ref{nf:two} und teils auch \ref{nf:three} sind hier zugehörig.
    Hier müssen algorithmische sowie technologische Verbesserungen in Leistung oder Laufzeit aufgezeigt und mögliche Lösungswege aufgezeigt oder implementiert werden.
    Besonders gewählte Datenbankformate werden hier thematisiert.

    
    