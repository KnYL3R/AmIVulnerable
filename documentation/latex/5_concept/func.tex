\subsection{Funktionale Anforderungen} \label{sec:Funktionale_Anforderungen}
    Bei der Umsetzung dieser Schwachstellenanalyse-\ac{API} ergeben sich nun verschidene funktionale Anforderungen aus den Forschungsfragen:
    \begin{enumerate}[label=\textbf{FRQ-\Roman*}, leftmargin=1cm]
        \item (\ref{q:one}, \ref{q:two}) Einladen und Konvertieren der Schwachstellendaten und persistente Speicherung dieser in einer internen Datenbank \label{f:one}
        \item (\ref{q:one}) Überprüfung von Paketen auf Sicherheitslücken mittels Abgleich zur internen Schwachstellendatenbank \label{f:two}
        \item (\ref{q:two}) Clonen eines Repositories von Github. \label{f:three}
        \item (\ref{q:two}) Überprüfung von allen Abhängigkeiten eines Repositories mittels Abgleich zur internen Schwachstellendatenbank \label{f:four}
        \item (\ref{q:two}) Extraktion und Rückgabe eines Abhängigkeitsbaums für alle Abhängigkeiten mit Sicherheits\-lücken \label{f:five}
        \item (\ref{q:five}) Aktualisierung der Schwachstellendatenbasis \label{f:six}
        \item (\ref{q:four}) Bereitstellung der \ac{API} in einem Container \label{f:seven}
    \end{enumerate}
    Diese Funktionalitäten setzten teils mehrere interne Abläufe voraus und beinhalten mehrere Endpunkte der \ac{API}.
