\subsubsection{ReposVul} \label{sec:ReposVul}
    \begin{description}
        \item[Adressierte Punkte]\hfill \\
            Im Paper \glqq ReposVul: A Repository-Level High-Quality Vulnerability Dataset\grqq\cite{article:wang2024reposvul} von Xinchen Wang et al. wird ein Datenbeschaffungs-Framework vorgestellt, welches Schwachstellendatensätze konstruiert.
            Notwendig ist diese Arbeit durch die immer weiter steigende Anzahl an Open-Source-Software sowie deren Sicherheitsauswirkungen.
            In dieser Arbeit wird sich verworrenen und veralteten Patches in bestehenden Schwachstellendatensätzen gewidmet.
            Ziel ist es, einen hochwertigen Schwachstellendatensatz herzustellen.
            Dieser soll für Schwachstellenerkennungsmodell später nutzbar sein.
        \item[Duchgeführte Maßnahmen]\hfill \\
            Um nun einen solchen Datensatz erstellen zu können wurde zuerst ein automatisiertes Datenbeschaffungs-Framework, \glqq ReposVul\grqq, erstellt.
            Dieses erstellte einen ersten Schwachstellendatensatz auf Repository-Level.
            Unter Verwendung von Large Language Models sowie statischen Codeanalysetools wurde zwischen Code-Änderungen und verworrenen Patches unterschieden.
            Veraltete Patches werden hier letztendlich durch ein implementiertes tracebasiertes Filtermodul erkannt.
            Weiterhin wurden interprozeduale Aufrufbeziehungen von Schwachstellen erfasst und extrahiert.
            Die resultierende Datensammlung enthält 6.134 CVE-Einträge mit 236 CWE-Typen, die die Programmiersprachen c++, C, Java sowie Python in 1.491 Projekten darstellen.
        \item[Ergebnisse]\hfill \\
            Es wurde ein hochwertiger Schwachstellendatensatz auf Repository-Level erstellt.
            Dieser verringert die Probleme, welche durch verworrene oder veraltete Patches in anderen Datensätzen entstehen.
            Es wird hier eine effektive Methode für Kennzeichnung und Erkennung von Schwachstellen sowie veralteten Patches bereitgestellt.
        \item[Unterschiede]\hfill \\
            Es wird in dieser Arbeit nicht auf die Entwicklung einer Softwarelösung, sondern dem erstellen einer Umfangreichen Datenbasis wert gelegt.
            Es werden weiterhin keine Möglichkeiten betrachtet, welche die Resultierenden Daten besser für Mensch und Maschine Nutzbar machen, wie zum Beispiel \ac{JSON-LD}.
            Außerdem ist es hier nicht möglich direkt ein einzelnen Repository zu analysieren.
    \end{description}