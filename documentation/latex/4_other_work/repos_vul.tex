\subsubsection{ReposVul} \label{sec:ReposVul}
    Im Paper \glqq ReposVul: A Repository-Level High-Quality Vulnerability Dataset\grqq~von Xinchen Wang et al. wird ein Datenbeschaffungs-Framework vorgestellt, welches auf Repository basis Schwachstellendatensätze konstruiert. %TODO:Quelle
    Hierbei wird zwischen Code-Änderungen und Patches zu Schwachstellen unterschieden, Aufrufbeziehungen von Schwachstellen extrahiert und ein Filtermodul für veraltete Patches bereitgestellt.
    Analysiert wurden unter anderem auch transitive Abhängigkeiten
    Es wurden in dieser Analyse C++, C, Java und Python als Programmiersprachen betrachtet und über 1491 Projekte ausgewertet.
    