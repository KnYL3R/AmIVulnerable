\subsubsection{ReposVul} \label{sec:ReposVul}
Im Paper \glqq ReposVul: A Repository-Level High-Quality Vulnerability Dataset\grqq~von Xinchen Wang et al. wird ein Datenbeschaffungs-Framework vorgestellt, welches auf Repositorie basis Schwachstellendatensätze konstruiert.
Hierbei wird zwischen Code-Änderungen und Patches zu Schwachstellen unterschieden, Aufrufbeziehungen von Schwachstellen extrahiert und ein Filtermodul für veraltete Patches bereitgestellt.
Analysiert wurden unter anderem auch transitive Abhängigkeiten
Es wurden in dieser Analyse C++, C, Java und Python als Programmiersprachen betrachtet und über 1491 Projekte ausgewertet.
% \\
% Als wesentlicher Unterschied zu dieser Arbeit ist, dass die Resultatdaten nicht als Abhängigkeitsbaum dargestellt sind.
% Weiterhin ist ein Framework und kein Servive, wie z.B. eine API, zur Verfügung gestellt und somit nicht direkt in Entwicklungsprozesse zu integrieren.
% Resultatdaten werden im JSON Lines-Format ausgegeben und nicht als standardisiertes, maschinenlesbares JSON-LD, womit sie nicht direkt weiterverwendbar sind.
