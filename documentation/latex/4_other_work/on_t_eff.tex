\subsubsection{Transitivity and Granularity on Vulnerability Propagation} \label{sec:Transitivity}
    \begin{description}
        \item[Adressierte Punkte]\hfill \\
            Im Paper \glqq On the Effect of Transitivity and Granularity on Vulnerability Propagation in the Maven Ecosystem\grqq\textsuperscript{\cite{article:OnTheEffect10123571}} von Amir M. Mir et al. wurden Schwachstellen im Maven-Ökosystem\textsuperscript{\cite{link:Maven}} durch eine Analyse von $3$ Millionen Maven-Paketen analysiert.
            Dabei wurde vor allem ein Augenmerk auf Granularität sowie Transitivität der Schwachstellen gelegt.
            Es wurde betrachtet, in wie fern sich Sicherheitslücken von direkten zu transitiven Abhängigkeiten propagieren.
        \item[Duchgeführte Maßnahmen]\hfill \\
            Aus den Daten einer Umfangreichen Analyse von Maven-Projekten auf Sicherheitslücken, direkte sowie auch transitive, wurden diese durch einen Wissensgraphen dargestellt.
            Gesamt beinhaltet die Studie $1.300$ Sicherheitsberichte, welche für die Identifizierung von Schwachstellen genutzt wurden.
        \item[Ergebnisse]\hfill \\
            Es wurde aufgezeigt, dass unter Betrachtung aller transitiven Abhängigkeiten etwa $31$ \% der Projekte Schwachstellen beinhalten.
            Die Granularitätsanalyse jedoch ergab, dass nur $1,2$ \% der schwachstellenbetroffenen Projekte auch wirklich schwachstellenbehafteten Code aus Abhängig\-keiten erreichen.
            Es wird hier vorgeschlagen durch die niedrige Anzahl an wirklich erreichbaren Schwachstellen, die Betrachtungstiefe von transitiven Abhängigkeiten in Schachstellenanalysen der Laufzeit und Ressourcen zu gute kommend zu begrenzen.
        \item[Unterschiede]\hfill \\
            In dieser Arbeit wird nicht direkt ein Tool bereitgestellt, mit welchem eine Schwachstellenanalyse durchgeführt werden kann.
            Weiterhin stellt sie aber die Frage, welche gefundenen Schwachstellen unter Betrachtung aller transitiven und direkten Abhängigkeiten wirklich Auswirkungen auf die Applikationssicherheit haben.
    \end{description}