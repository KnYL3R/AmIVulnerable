\section{Andere Arbeiten} \label{sec:Andere}
    Vor der Konzeption der Softwarelösung einer Dependency-\ac{API} muss erst ein Blick auf die bisherige Forschung und Entwicklung in diesem Bereich geworfen werden.
    Es gibt bereits verschiedene Applikationen, \ac{API}'s und Plattformen, die sich mit der Herausforderung beschäftigen, Abhängigkeiten in Softwarelösungen zu verwalten und potenzielle Sicherheitsrisiken zu identifizieren.
    Folgend werden einige der bekanntesten Softwarelösungen betrachtet, die für diese Zwecke entwickelt wurden.
    Es soll ein Überblick über den aktuellen Stand der Technik und mögliche Herausforderungen entstehen um Herausforderungen, denen sich Entwickler bei der Verwaltung von Abhängigkeiten und der Gewährleistung der Sicherheit in ihren Projekten gegenübersehen, besser einzuschätzen.
    \subsection{Softwaretools} \label{subsec:Softwaretools}
    \subsubsection{NIST-API} \label{subsubsec:NIST_API}
    NIST-API
    \subsection{Github Dependa Bot} \label{sec:Dependa}

    \subsubsection{Snyk} \label{sec:Snyk}
Die Synk Plattform ist ein weiteres Tool, mit welchem Sicherheitsschwachstellen in Applikationen gemanagt werden können.
Es werden betroffene Abhängigkeiten identifiziert und Lösungen angeboten diese Schwachstellen zu schließen.

Es werden verschiedene Teile der Softwareentwicklung von Synk betrachtet darunter der Code selbst, Container und die Infrastruktur.
\glqq Synk Open Source\grqq~erkennt Schwachstellen in \glqq Open Source\grqq-Abhängigkeiten und \glqq Synk Code\grqq~erkennt diese im Code selbst.
\glqq Synk Container\grqq~erkennt Schwachstellen in Container-Images sowie Kubernetesanwendungen.
\glqq Synk Infrastructure as Code (IaC)\grqq~erkennt Fehlkonfigurationen in Terraform, CloudFormation, Kubernetes und Azure-Formlagen.

Durch Synk werden nicht nur Meldungen generiert, welche Abhängigkeiten Schwachstellen aufweisen sondern in der gesammten Deployment-Kette auf Schwachstellen geachtet.
Lösungen bzw. Alternativen werden angeboten und zusätztlich kann dieses Tool auch in die CI/CD eingebunden werden um einen sicherheitskritischen Bau der Applikation zu verhindern.   
    \subsubsection{OWASP Dependency-Check} \label{sec:OWASP-Dependency-Check}
    Das Software-Composition-Analysis-Tool Dependency Check von der OWASP Foundation analysiert die Codebasis auf bekannte Schwachstellen.
    Dabei werden auf Common-Platform-Enumeration-Kennungen (CPE) für die genutzten Abhängigkeiten geprüft und falls vorhanden ein Bericht mit zugehöriger \ac{CVE}-Nummer erstellt.

    Die durch die automatischen Analysen erstellten Berichte enthalten nicht nur die jeweiligen Schwachstelle selbst, sondern auch Maßnahmen zum Schließen jener.
    \newpage
    \subsection{Wissenschaftliche Arbeiten} \label{subsec:Wiss_Arbeiten}
    \subsubsection{ReposVul} \label{sec:ReposVul}
    \begin{description}
        \item[Adressierte Punkte]\hfill \\
            Im Paper \glqq ReposVul: A Repository-Level High-Quality Vulnerability Dataset\grqq\textsuperscript{\cite{article:wang2024reposvul}} von Xinchen Wang et al. wird ein Datenbeschaffungs-Framework vorgestellt, welches Schwachstellendatensätze konstruiert.
            Notwendig ist diese Arbeit durch die immer weiter steigende Anzahl an Open-Source-Software sowie deren Sicherheitsauswirkungen.
            In dieser Arbeit wird sich verworrenen und veralteten Patches in bestehenden Schwachstellendatensätzen gewidmet.
            Ziel ist es, einen hochwertigen Schwachstellendatensatz herzustellen.
            Dieser soll für Schwachstellenerkennungsmodell später nutzbar sein.
        \item[Duchgeführte Maßnahmen]\hfill \\
            Um nun einen solchen Datensatz erstellen zu können wurde zuerst ein automatisiertes Daten\-be\-schaffungs-Framework, \glqq ReposVul\grqq, erstellt.
            Dieses erstellte einen ersten Schwachstellendatensatz auf Repository-Level.
            Unter Verwendung von Large Language Models sowie statischen Codeanalysetools wurde zwischen Code-Änderungen und verworrenen Patches unterschieden.
            Veraltete Patches werden hier letztendlich durch ein implementiertes tracebasiertes Filtermodul erkannt.
            Weiterhin wurden interprozeduale Aufrufbeziehungen von Schwachstellen erfasst und extrahiert.
            Die resultierende Datensammlung enthält $6.134$ CVE-Einträge mit $236$ CWE-Typen, die die Programmiersprachen c++, C, Java sowie Python in 1.491 Projekten darstellen.
        \item[Ergebnisse]\hfill \\
            Es wurde ein hochwertiger Schwachstellendatensatz auf Repository-Level erstellt.
            Dieser verringert die Probleme, welche durch verworrene oder veraltete Patches in anderen Datensätzen entstehen.
            Es wird hier eine effektive Methode für Kennzeichnung und Erkennung von Schwachstellen sowie veralteten Patches bereitgestellt.
        \item[Unterschiede]\hfill \\
            Es wird in dieser Arbeit nicht auf die Entwicklung einer Softwarelösung, sondern dem erstellen einer Umfangreichen Datenbasis wert gelegt.
            Es werden weiterhin keine Möglichkeiten betrachtet, welche die Resultierenden Daten besser für Mensch und Maschine Nutzbar machen, wie zum Beispiel \ac{JSON-LD}.
            Außerdem ist es hier nicht möglich direkt ein einzelnen Repository zu analysieren.
    \end{description}
    \subsubsection{Transitivity and Granularity on Vulnerability Propagation} \label{sec:Transitivity}
    \begin{description}
        \item[Adressierte Punkte]\hfill \\
            Im Paper \glqq On the Effect of Transitivity and Granularity on Vulnerability Propagation in the Maven Ecosystem\grqq\textsuperscript{\cite{article:OnTheEffect10123571}} von Amir M. Mir et al. wurden Schwachstellen im Maven-Ökosystem\textsuperscript{\cite{link:Maven}} durch eine Analyse von $3$ Millionen Maven-Paketen analysiert.
            Dabei wurde vor allem ein Augenmerk auf Granularität sowie Transitivität der Schwachstellen gelegt.
            Es wurde betrachtet, in wie fern sich Sicherheitslücken von direkten zu transitiven Abhängigkeiten propagieren.
        \item[Duchgeführte Maßnahmen]\hfill \\
            Aus den Daten einer Umfangreichen Analyse von Maven-Projekten auf Sicherheitslücken, direkte sowie auch transitive, wurden diese durch einen Wissensgraphen dargestellt.
            Gesamt beinhaltet die Studie $1.300$ Sicherheitsberichte, welche für die Identifizierung von Schwachstellen genutzt wurden.
        \item[Ergebnisse]\hfill \\
            Es wurde aufgezeigt, dass unter Betrachtung aller transitiven Abhängigkeiten etwa $31$ \% der Projekte Schwachstellen beinhalten.
            Die Granularitätsanalyse jedoch ergab, dass nur $1,2$ \% der schwachstellenbetroffenen Projekte auch wirklich schwachstellenbehafteten Code aus Abhängig\-keiten erreichen.
            Es wird hier vorgeschlagen durch die niedrige Anzahl an wirklich erreichbaren Schwachstellen, die Betrachtungstiefe von transitiven Abhängigkeiten in Schachstellenanalysen der Laufzeit und Ressourcen zu gute kommend zu begrenzen.
        \item[Unterschiede]\hfill \\
            In dieser Arbeit wird nicht direkt ein Tool bereitgestellt, mit welchem eine Schwachstellenanalyse durchgeführt werden kann.
            Weiterhin stellt sie aber die Frage, welche gefundenen Schwachstellen unter Betrachtung aller transitiven und direkten Abhängigkeiten wirklich Auswirkungen auf die Applikationssicherheit haben.
    \end{description}
    \subsubsection{Demystifying Vulnerability Propagation} \label{sec:DVP}
Im Paper \glqq Demystifying the Vulnerability Propagation and Its Evolution via Dependency Trees in the NPM Ecosystem\grqq~von Chengwei Liu et al. wurde eine ausführliche Studie zum NPM-Ökosystem durchgeführt.
Betrachtet wurden auch Versionierungen von Paketen.
Es wurde ein kompletter Abhängigkeits-Wissensgraph des NPM-Ökosystems aufgestellt, welcher dann zum identifizieren von Schwachstellen sowie deren Propagation genutzt wurde.
Das entwickelte \glqq DVReme\grqq~Tool verfolgt Pakete und ihre Abhängigkeiten zurück und sucht diese nach Schwachstellen ab.
Ergebnis der Studie war, dass 20\% der Bibliotheken im NPM-Ökosystem Schachstellen aufweisen.
30\% dieser Bibliotheken enthalten Schwachstellen aus direkten Abhängigkeiten.
Ein weiterer Fund war, dass je länger eine bekannte Schachstelle offen bleibt desto mehr Pakete sind im gesamten Abhängigkeitsbaum des NPM-Ökosystems betroffen davon. 
    Zusammenfassend ist zu sehen, dass in den erwähnten Arbeiten kein Service vorhanden ist, welcher unabhängig von Programmiersprache ein Paket, mehrere Pakete sowie ganze Repositories auf Sicherheitslücken mittels CVE-Daten untersucht wobei Abhängigkeitsbäume bei der Repository-Analyse hineinbezogen werden und diese Daten dann als weiterverwendbare Daten zum Beispiel als JSON-LD durch einen API-Endpunkt bereitgestellt sind.
    Dieser Service soll selbst nicht abhängig von einen zu analysierenden Repository sein, sondern eigens verschiedene Repositories durch Angabe dieser analysieren. 