\section{Andere Arbeiten} \label{sec:Andere}
    Vor der Konzeption der Softwarelösung einer Dependency-\ac{API} muss erst ein Blick auf die bisherige Forschung und Entwicklung in diesem Bereich geworfen werden.
    Es gibt bereits verschiedene Applikationen, \ac{API}'s und Plattformen, die sich mit der Herausforderung beschäftigen, Abhängigkeiten in Softwarelösungen zu verwalten und potenzielle Sicherheitsrisiken zu identifizieren.
    Folgend werden einige der bekanntesten Softwarelösungen betrachtet, die für diese Zwecke entwickelt wurden.
    Es soll ein Überblick über den aktuellen Stand der Technik und mögliche Herausforderungen entstehen um Herausforderungen, denen sich Entwickler bei der Verwaltung von Abhängigkeiten und der Gewährleistung der Sicherheit in ihren Projekten gegenübersehen, besser einzuschätzen.
    \subsection{Softwaretools} \label{subsec:Softwaretools}
    \subsubsection{\ac{NIST}-\ac{API}} \label{subsubsec:NIST_API}
    \ac{NIST}-\ac{API}
    \subsection{Github Dependa Bot} \label{sec:Dependa}

    \subsection{Snyk} \label{sec:Snyk}
https://docs.github.com/en/code-security/getting-started/github-security-features
    \subsubsection{OWASP Dependency-Check} \label{sec:OWASP-Dependency-Check}
Das Software-Composition-Analysis-Tool Dependency Check von der OWASP Foundation analysiert die Codebasis auf bekannte Schwachstellen.
Dabei werden auf Common-Platform-Enumeration-Kennungen (CPE) für die genutzten Abhängigkeiten geprüft und falls vorhanden ein Bericht mit zugehöriger \ac{CVE}-Nummer erstellt.
\\ \\
Die durch die automatischen Analysen erstellten Berichte enthalten nicht nur die jeweiligen Schwachstelle selbst, sondern auch Maßnahmen zum Schließen jener.
    \newpage
    \subsection{Wissenschaftliche Arbeiten} \label{subsec:Wiss_Arbeiten}
    \subsubsection{ReposVul} \label{sec:ReposVul}
    Im Paper \glqq ReposVul: A Repository-Level High-Quality Vulnerability Dataset\grqq~von Xinchen Wang et al. wird ein Datenbeschaffungs-Framework vorgestellt, welches auf Repository basis Schwachstellendatensätze konstruiert. %TODO:Quelle
    Hierbei wird zwischen Code-Änderungen und Patches zu Schwachstellen unterschieden, Aufrufbeziehungen von Schwachstellen extrahiert und ein Filtermodul für veraltete Patches bereitgestellt.
    Analysiert wurden unter anderem auch transitive Abhängigkeiten
    Es wurden in dieser Analyse C++, C, Java und Python als Programmiersprachen betrachtet und über 1491 Projekte ausgewertet.
    
    \subsubsection{Transitivity and Granularity on Vulnerability Propagation} \label{sec:Transitivity}
Im Paper \glqq On the Effect of Transitivity and Granularity on Vulnerability Propagation in the Maven Ecosystem\grqq~von Amir M. Mir et al. wurden Schwachstellen im Maven-Ökosystem durch eine Analyse von 3 Millionen Maven-Paketen analysiert.
Dabei wurde auf direkte sowie transitive Abhängigkeiten geachtet und die Verteilung von Sicherheitslücken im Datensatz betrachtet.
Analysiert wurden weiterhin die Erreichbarkeit der Sicherheitslücken mit dem Ergebnis, dass nur etwa 1\% der transitiven Abhängigkeiten mit Sicher\-heits\-lücken diese ausweisen.
Dazu wurde eine Datenverarbeitungspipeline implementiert, wobei der Abhängigkeitsdatensatz in einem Wissensgraph dargestellt und analsiert wurde.
    \subsubsection{Demystifying Vulnerability Propagation} \label{sec:DVP}
Im Paper \glqq Demystifying the Vulnerability Propagation and Its Evolution via Dependency Trees in the NPM Ecosystem\grqq~von Chengwei Liu et al. wurde eine ausführliche Studie zum NPM-Ökosystem durchgeführt.
Betrachtet wurden auch Versionierungen von Paketen.
Es wurde ein kompletter Abhängigkeits-Wissensgraph des NPM-Ökosystems aufgestellt, welcher dann zum identifizieren von Schwachstellen sowie deren Propagation genutzt wurde.
Das entwickelte \glqq DVReme\grqq~Tool verfolgt Pakete und ihre Abhängigkeiten zurück und sucht diese nach Schwachstellen ab.
Ergebnis der Studie war, dass 20\% der Bibliotheken im NPM-Ökosystem Schachstellen aufweisen.
30\% dieser Bibliotheken enthalten Schwachstellen aus direkten Abhängigkeiten.
Ein weiterer Fund war, dass je länger eine bekannte Schachstelle offen bleibt desto mehr Pakete sind im gesamten Abhängigkeitsbaum des NPM-Ökosystems betroffen davon. 
    Zusammenfassend ist zu sehen, dass in den erwähnten Arbeiten kein Service vorhanden ist, welcher unabhängig von Programmiersprache ein Paket, mehrere Pakete sowie ganze Repositories auf Sicherheitslücken mittels \ac{CVE}-Daten untersucht wobei Abhängigkeitsbäume bei der Repository-Analyse hineinbezogen werden und diese Daten dann als weiterverwendbare Daten zum Beispiel als \ac{JSON-LD} durch einen \ac{API}-Endpunkt bereitgestellt sind.
    Dieser Service soll selbst nicht abhängig von einen zu analysierenden Repository sein, sondern eigens verschiedene Repositories durch Angabe dieser analysieren.
