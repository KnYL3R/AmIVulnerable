\subsubsection{Demystifying Vulnerability Propagation} \label{sec:DVP}
    \begin{description}
        \item[Adressierte Punkte]\hfill \\
            Im Paper \glqq Demystifying the Vulnerability Propagation and Its Evolution via Dependency Trees in the NPM Ecosystem\grqq\textsuperscript{\cite{article:Liu2022DemystifyingTV}} von Chengwei Liu et al. wurde eine ausführliche Studie zum NPM-Ökosystem durchgeführt.
            In dieser Studie wurde die Schwachstellenausbreitung und -entwicklung betrachtet.
            Diese Schwachstellen wurden in Abhängigkeitsbäumen Identifiziert.
            Aus diesen Daten wurden Lösungen zur Schwachstellenbehebung entwickelt.
        \item[Duchgeführte Maßnahmen]\hfill \\
            In dieser Arbeit wurden Abhängigkeitsbäume, auch Dependency-Graphen (DVGraph) genannt, konstruiert.
            Es wurde weiterhin ein DVResolver-Algorithmus, welcher Abhängigkeitsbäume auflöst und ein DVReme-Tool entwickelt, welches der Schwachstellenbehebung dient.
            Es wurde hier JavaScript für die Analyse der NPM-Pakete und Abhängigkeitsbäume genutzt.
            Schachstellen wurden dabei durch die Verwendung von \ac{CVE}-Daten erkannt. 
        \item[Ergebnisse]\hfill \\
            Es wurden durch diese Studie weit verbreitete Schwachstellen in NPM-Paketen offen gelegt sowie ein Tool zur effektiven Bekämpfung dieses implementiert.
            Dabei hat das entwickelte DVReme-Tool sogar eine bessere Leistung als das offizielle npm audit fix-Tool.
            Ergebnis der Studie war, dass 20 \% der Bibliotheken im NPM-Ökosystem Schwachstellen aufweisen.
            30 \% dieser Bibliotheken enthalten Schwachstellen aus direkten Abhängigkeiten.
            Aus den erhaltenen Daten wurden Empfehlungen zur Verbesserung der Sicherheit im NPM-Ökosystem zusammengefasst.
        \item[Unterschiede]\hfill \\
            Im gegensatz zu dieser Arbeit wird sich hier auf das NPM-Ökosystem festgelegt.
            Das entstandene Tool ist auch nur für diese geeignet und bietet in der Hinsicht auch keine Erweiterungsmöglichkeit auf weitere Programmiersprachen oder Frameworks.
            Weiterhin sind die entstandenen nicht in einem Standardisierten Format für den Nutzer zu erhalten. 
    \end{description}