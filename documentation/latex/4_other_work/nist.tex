\subsubsection{NIST-API} \label{sec:NIST-API}
    Das \glqq \ac{NIST}\grqq~bietet zwei \ac{CVE}-\ac{API}'s an.\textsuperscript{\cite{link:NISTAPI}}
    Diese sind die \ac{CVE}-\ac{API}\textsuperscript{\cite{link:NISTAPI}} und die \ac{CVE}-Change-History-\ac{API}\textsuperscript{\cite{link:NISTAPI}}.

    Die \textbf{\ac{CVE}-\ac{API}} nimmt Anfragen zu einer oder mehreren \ac{CVE}-Einträgen entgegen und gibt diese einschließlich Details zu bekannten Sicherheitsanfälligkeiten wie Beschreibungen, betroffenen Softwareprodukten und Schweregraden zurück.
    Um nun aktuelle \ac{CVE}-Daten abzurufen und zu Nutzen muss lediglich diese \ac{API} in eigene Anwendungen oder Tools zu integriert werden.
    Dies ermöglicht es, automatisierte Prozesse zur Überprüfung von Abhängigkeiten auf Sicherheitslücken zu implementieren und schnell auf neu entdeckte \ac{CVE}'s zu reagieren.
    Durch das Vorhandensein von mehr als 240.000 Einträgen wird hier eine offset-basierte Paginierung der Daten durchgeführt um Anfragen für große Sammlungen zu beantworten.

    Die \textbf{\ac{CVE}-Change-History-\ac{API}} bietet einen historischen Überblick über Änderungen an \ac{CVE}-Einträgen.
    Durch diese können Änderungsverläufe und Aktualisierungen von \ac{CVE}'s besser nachvollzogen werden.
    Aus den dadurch bereitgestellten Daten können Muster oder Trends bei Sicherheitsvorfällen identifiziert werden und entsprechende Vorkehrungen seitens der Entwickler getroffen werden.

    Durch eine Nutzung dieser \ac{NIST}-\ac{API}s können Entwickler und Sicherheitspersonal auf aktuelle \ac{CVE}-Daten zugreifen.
    Die Sicherheit ihrer Softwarelösungen kann verbessert und potenzielle Risiken minimiert werden.
    Im Vergleich zu anderen noch vorzustellenden Lösungen muss hier eine Applikation angebunden werden, die die Daten grafisch aufbereitet, da diese nur in einem JSON-Format zurückgegeben werden. 
    Die Integration dieser Daten in Sicherheitstools und Plattformen trägt dazu bei, die Effektivität von Sicherheitsmaßnahmen in der Softwareentwicklung zu erhöhen und eine proaktive Sicherheitsstrategie zu fördern.
